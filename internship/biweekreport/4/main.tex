%-------------------------------------------------------------------------------%	PACKAGES AND OTHER DOCUMENT CONFIGURATIONS
%-------------------------------------------------------------------------------
\documentclass[12pt]{article}
%\usepackage[francais]{babel}
\usepackage[T1]{fontenc}
\usepackage[utf8]{inputenc}
\usepackage[top=3cm, bottom=3cm, left=3cm, right=3cm]{geometry}
\usepackage{fancyhdr}
\usepackage{lastpage}
\usepackage{graphicx}
\usepackage[colorinlistoftodos]{todonotes}
\usepackage{amsmath}
\usepackage{color}
\usepackage{array}
\definecolor{light-gray}{gray}{0.95}
\usepackage{xcolor}
\usepackage{listings}
\setlength{\parindent}{0pt}

\usepackage{path}
\usepackage[bookmarks,bookmarksnumbered,colorlinks,hyperindex,pagebackref,urlcolor=blue,final]{hyperref}

\definecolor{email}{rgb}{0,.5,0}

\definecolor{codegreen}{rgb}{0,0.6,0}
\definecolor{codegray}{rgb}{0.5,0.5,0.5}
\definecolor{codepurple}{rgb}{0.58,0,0.82}
\definecolor{backcolour}{rgb}{0.95,0.95,0.92}


\begin{document}
\begin{titlepage}

\def\EMAILArabella{Arabella.Brayer@ulb.ac.be}

\newcommand{\imagepath}{../img}
\newcommand{\university}{Université Libre de Bruxelles}
\newcommand{\faculty}{Département d'informatique}
\newcommand{\course}{TRAN-F-501 - Internship}
\newcommand{\authors}{\bfseries{Arabella \textsc{Brayer}} -- \emailArabella}
\newcommand{\supervisor}{Paul \textsc{Rodriguez}}
\newcommand{\HRule}{\rule{\linewidth}{0.5mm}} % Defines a new command for the horizontal lines, change thickness here
\newcommand{\emailArabella}{\href{mailto:\EMAILArabella?subject=[TRAN-F-501 - Internship]}{\color{\EMAILArabella}\path|\EMAILArabella|}}
\center % Center everything on the page
 
%-------------------------------------------------------------------------------
%	HEADING SECTIONS
%-------------------------------------------------------------------------------

\textsc{\LARGE \university}\\[1.5cm] % Name of your university/college
\textsc{\Large \faculty}\\[0.5cm] % Major heading such as course name
\textsc{\large \course}\\[0.5cm] % Minor heading such as course title

%-------------------------------------------------------------------------------%	TITLE SECTION
%-------------------------------------------------------------------------------\HRule \\[0.4cm]
{ \huge \bfseries Implementation of a file system service on the HIPPEROS micro-kernel}\\[0.4cm] % Title of your document
\HRule \\[1.5cm]
 
%-------------------------------------------------------------------------------%	AUTHOR SECTION
%-------------------------------------------------------------------------------
\begin{minipage}{0.4\textwidth}
\begin{flushleft} \large
\emph{Authors:}\\
Arabella \textsc{Brayer} -- \href{mailto:Arabella.Brayer@ulb.ac.be?subject=[TRAN-F-501 - Internship]}{email}
\end{flushleft}
\end{minipage}
~
\begin{minipage}{0.4\textwidth}
\begin{flushright} \large
\emph{Supervised by:} \\
\supervisor % Supervisor's Name
\end{flushright}
\end{minipage}\\[2cm]


%----------------------------------------------------------------------------------------
%	DATE SECTION
%----------------------------------------------------------------------------------------

{\large September, 25, 2017 }\\[2cm] % Date, change the \today to a set date if you want to be precise


%----------------------------------------------------------------------------------------
%	LOGO SECTION
%----------------------------------------------------------------------------------------

\includegraphics{\imagepath/logoULB.png}\\[1cm] 
 
%----------------------------------------------------------------------------------------

\vfill % Fill the rest of the page with whitespace

\end{titlepage}

\newpage


\pagestyle{fancy}
\fancyhf{}
\setlength\headheight{15pt}
\fancyhead[L]{Bi-weekly report}
\fancyhead[R]{}
\fancyfoot[L]{September, 25, 2017}
\fancyfoot[R]{\thepage}

\newpage

%\tableofcontents
%-----------------------------------------------------------
%\newpage
\section{Presentation}
\subsection{Recall of the subject}
The aim of the internship is to develop and integrate a new service for the real-time operating system HIPPEROS, 
that would provide the functionality of the FATFS library to users.
Currently, HIPPEROS uses a FATFS library directly, the aim is to introduce an indirection in order to 
avoid concurrent access that might occur.

\subsection{Current task}
The tasks during the first six weeks were summarized in the mid-term 
evaluation. Generally speaking, the first part of the 
internship consisted in the implementation of 
the service as well as benchmarks of Open and Read operations. 
This aim is now reached and the next tasks are to implement the Write and 
Close operations, and also benchmarks.

The current task is still to enhance the implementation of the prototype 
in order to demonstrate the correctness of the operation and to test 
the efficiency in comparison with the direct call of the FATFS library. 

As a remainder, two weeks ago, I was waiting for a review for the library
integration and working on the benchmarks. These tasks are now finished, 
the library integration has been accepted and the benchmarks are now in 
review.

\section{Flow description}
\subsection{First week 11/09 to 15/09}
The first week, I have been working on the review for the library integration, 
and working on the improvement of the benchmarks. 
The second part of the week, I have been working back on the write function. 
On Friday, the PR about the benchmark had made good progress and 
the write and close operations were basically implemented.

\subsection{Second week 18/09 to 22/09}

\subsection{Next goal}

At this time, the last PR on the library integration has not been reviewed.
For the next week, the idea is to finish the benchmarks and start the work around the 
write function even if it has not yet been done.

\end{document}
