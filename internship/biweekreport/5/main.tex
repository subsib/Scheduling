%-------------------------------------------------------------------------------%	PACKAGES AND OTHER DOCUMENT CONFIGURATIONS
%-------------------------------------------------------------------------------
\documentclass[12pt]{article}
%\usepackage[francais]{babel}
\usepackage[T1]{fontenc}
\usepackage[utf8]{inputenc}
\usepackage[top=3cm, bottom=3cm, left=3cm, right=3cm]{geometry}
\usepackage{fancyhdr}
\usepackage{lastpage}
\usepackage{graphicx}
\usepackage[colorinlistoftodos]{todonotes}
\usepackage{amsmath}
\usepackage{color}
\usepackage{array}
\definecolor{light-gray}{gray}{0.95}
\usepackage{xcolor}
\usepackage{listings}
\setlength{\parindent}{0pt}

\usepackage{path}
\usepackage[bookmarks,bookmarksnumbered,colorlinks,hyperindex,pagebackref,urlcolor=blue,final]{hyperref}

\definecolor{email}{rgb}{0,.5,0}

\definecolor{codegreen}{rgb}{0,0.6,0}
\definecolor{codegray}{rgb}{0.5,0.5,0.5}
\definecolor{codepurple}{rgb}{0.58,0,0.82}
\definecolor{backcolour}{rgb}{0.95,0.95,0.92}


\begin{document}
\begin{titlepage}

\def\EMAILArabella{Arabella.Brayer@ulb.ac.be}

\newcommand{\imagepath}{../img}
\newcommand{\university}{Université Libre de Bruxelles}
\newcommand{\faculty}{Département d'informatique}
\newcommand{\course}{TRAN-F-501 - Internship}
\newcommand{\authors}{\bfseries{Arabella \textsc{Brayer}} -- \emailArabella}
\newcommand{\supervisor}{Paul \textsc{Rodriguez}}
\newcommand{\HRule}{\rule{\linewidth}{0.5mm}} % Defines a new command for the horizontal lines, change thickness here
\newcommand{\emailArabella}{\href{mailto:\EMAILArabella?subject=[TRAN-F-501 - Internship]}{\color{\EMAILArabella}\path|\EMAILArabella|}}
\center % Center everything on the page
 
%-------------------------------------------------------------------------------
%	HEADING SECTIONS
%-------------------------------------------------------------------------------

\textsc{\LARGE \university}\\[1.5cm] % Name of your university/college
\textsc{\Large \faculty}\\[0.5cm] % Major heading such as course name
\textsc{\large \course}\\[0.5cm] % Minor heading such as course title

%-------------------------------------------------------------------------------%	TITLE SECTION
%-------------------------------------------------------------------------------\HRule \\[0.4cm]
{ \huge \bfseries Implementation of a file system service on the HIPPEROS micro-kernel}\\[0.4cm] % Title of your document
\HRule \\[1.5cm]
 
%-------------------------------------------------------------------------------%	AUTHOR SECTION
%-------------------------------------------------------------------------------
\begin{minipage}{0.4\textwidth}
\begin{flushleft} \large
\emph{Authors:}\\
Arabella \textsc{Brayer} -- \href{mailto:Arabella.Brayer@ulb.ac.be?subject=[TRAN-F-501 - Internship]}{email}
\end{flushleft}
\end{minipage}
~
\begin{minipage}{0.4\textwidth}
\begin{flushright} \large
\emph{Supervised by:} \\
\supervisor % Supervisor's Name
\end{flushright}
\end{minipage}\\[2cm]


%----------------------------------------------------------------------------------------
%	DATE SECTION
%----------------------------------------------------------------------------------------

{\large October, 09, 2017 }\\[2cm] % Date, change the \today to a set date if you want to be precise


%----------------------------------------------------------------------------------------
%	LOGO SECTION
%----------------------------------------------------------------------------------------

\includegraphics{\imagepath/logoULB.png}\\[1cm] 
 
%----------------------------------------------------------------------------------------

\vfill % Fill the rest of the page with whitespace

\end{titlepage}

\newpage


\pagestyle{fancy}
\fancyhf{}
\setlength\headheight{15pt}
\fancyhead[L]{Bi-weekly report}
\fancyhead[R]{}
\fancyfoot[L]{October, 9, 2017}
\fancyfoot[R]{\thepage}

\newpage

%\tableofcontents
%-----------------------------------------------------------
%\newpage
\section{Presentation}
\subsection{Recall of the subject}
The aim of the internship is to develop and integrate a new file system 
service for the real-time operating system HIPPEROS, 
that would provide the functionality of the libC library to users.
Currently, HIPPEROS uses a FATFS library directly, the aim is to introduce an indirection in order to 
avoid concurrent access that might occur.

\subsection{Current task}
The current task has a little bit changed since the last report.
Indeed, last time, I was working on the write function, and it was planed 
I implement benchmarks for that.

The current task is now to modify the file system service in order to 
make it POSIX compatible. This task has been determined on the Tuesday of the 
first week.

\section{Flow description}
\subsection{First week 25/09 to 29/09}
I first analyzed the POSIX requirements and stopped to work on the write 
benchmarks. I also had to document myself on newlib. 
After the meeting, on Tuesday, I worked on the refactoring of the 
current implementation. 
This consists of adding a user management in the service (with a list) 
as well as adapting all the calls and interactions.


\subsection{Second week 02/10 to 06/10}
The task of the start of the second week has been to add a managment of 
FIL list in the service. This is a grat change in the design.
During the first middle of the week, I have been working on that task, and 
also fixing some bugs.

During the second part of the week, I have been working on the adaptation 
of the previous benchmarks, and tests because the interfaces have changed.

\subsection{Next goal}
The next goal is to call directly \textit{hf\_open}, \textit{hf\_read},
\textit{hf\_write}, and \textit{hf\_close} by indirection calling 
\textit{fopen}, \textit{fread}, \textit{fwrite} and \textit{fclose} using 
POSIX. The library Newlib must be adapted to redict calls on the service.
Then, the Pull Request must be open to validate that great change.

\end{document}
