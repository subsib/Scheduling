%----------------------------------------------------------------------------------------
%	PACKAGES AND OTHER DOCUMENT CONFIGURATIONS
%----------------------------------------------------------------------------------------

\documentclass[12pt]{article}
%\usepackage[francais]{babel}
\usepackage[T1]{fontenc}
\usepackage[utf8]{inputenc}
\usepackage[top=3cm, bottom=3cm, left=3cm, right=3cm]{geometry}
\usepackage{fancyhdr}
\usepackage{lastpage}
\usepackage{graphicx}
\usepackage[colorinlistoftodos]{todonotes}
\usepackage{amsmath}
\usepackage{color}
\usepackage{array}
\definecolor{light-gray}{gray}{0.95}
\usepackage{xcolor}
\usepackage{listings}
\setlength{\parindent}{0pt}

\usepackage{path}
\usepackage[bookmarks,bookmarksnumbered,colorlinks,hyperindex,pagebackref,urlcolor=blue,final]{hyperref}

\definecolor{email}{rgb}{0,.5,0}

\definecolor{codegreen}{rgb}{0,0.6,0}
\definecolor{codegray}{rgb}{0.5,0.5,0.5}
\definecolor{codepurple}{rgb}{0.58,0,0.82}
\definecolor{backcolour}{rgb}{0.95,0.95,0.92}
 
% \lstdefinestyle{mystyle}{
%     backgroundcolor=\color{backcolour},   
%     commentstyle=\color{codegreen},
%     keywordstyle=\color{magenta},
%     numberstyle=\tiny\color{codegray},
%     stringstyle=\color{codepurple},
%     basicstyle=\footnotesize,
%     breakatwhitespace=false,         
%     breaklines=true,                 
%     captionpos=b,                    
%     keepspaces=true,                 
%     numbers=left,                    
%     numbersep=5pt,                  
%     showspaces=false,                
%     showstringspaces=false,
%     showtabs=false,                  
%     tabsize=2
% }
% \lstset{style=mystyle}

\begin{document}
\begin{titlepage}

\def\EMAILArabella{Arabella.Brayer@ulb.ac.be}

\newcommand{\imagepath}{../img}
\newcommand{\university}{Université Libre de Bruxelles}
\newcommand{\faculty}{Département d'informatique}
\newcommand{\course}{TRAN-F-501 - Internship}
\newcommand{\authors}{\bfseries{Arabella \textsc{Brayer}} -- \emailArabella}
\newcommand{\supervisor}{Paul \textsc{Rodriguez}}
\newcommand{\HRule}{\rule{\linewidth}{0.5mm}} % Defines a new command for the horizontal lines, change thickness here
\newcommand{\emailArabella}{\href{mailto:\EMAILArabella?subject=[TRAN-F-501 - Internship]}{\color{\EMAILArabella}\path|\EMAILArabella|}}
\center % Center everything on the page
 
%----------------------------------------------------------------------------------------
%	HEADING SECTIONS
%----------------------------------------------------------------------------------------

\textsc{\LARGE \university}\\[1.5cm] % Name of your university/college
\textsc{\Large \faculty}\\[0.5cm] % Major heading such as course name
\textsc{\large \course}\\[0.5cm] % Minor heading such as course title

%----------------------------------------------------------------------------------------
%	TITLE SECTION
%----------------------------------------------------------------------------------------
\HRule \\[0.4cm]
{ \huge \bfseries Implementation of a file system service on the HIPPEROS micro-kernel}\\[0.4cm] % Title of your document
\HRule \\[1.5cm]
 
%----------------------------------------------------------------------------------------
%	AUTHOR SECTION
%----------------------------------------------------------------------------------------

\begin{minipage}{0.4\textwidth}
\begin{flushleft} \large
\emph{Authors:}\\
Arabella \textsc{Brayer} -- \href{mailto:Arabella.Brayer@ulb.ac.be?subject=[TRAN-F-501 - Internship]}{email}
\end{flushleft}
\end{minipage}
~
\begin{minipage}{0.4\textwidth}
\begin{flushright} \large
\emph{Supervised by:} \\
\supervisor % Supervisor's Name
\end{flushright}
\end{minipage}\\[2cm]


%----------------------------------------------------------------------------------------
%	DATE SECTION
%----------------------------------------------------------------------------------------

{\large August, 12, 2017 }\\[2cm] % Date, change the \today to a set date if you want to be precise


%----------------------------------------------------------------------------------------
%	LOGO SECTION
%----------------------------------------------------------------------------------------

\includegraphics{\imagepath/logoULB.png}\\[1cm] 
 
%----------------------------------------------------------------------------------------

\vfill % Fill the rest of the page with whitespace

\end{titlepage}

\newpage


\pagestyle{fancy}
\fancyhf{}
\setlength\headheight{15pt}
\fancyhead[L]{Bi-weekly report}
\fancyhead[R]{}
\fancyfoot[L]{\today}
\fancyfoot[R]{\thepage}

\newpage

\tableofcontents
%-----------------------------------------------------------
\newpage
\section{Presentation}
\subsection{Recall of the subject}
The aim of the internship is to develop and integrate a new service for the real-time operating system HIPPEROS, 
that would provide the functionality of the FATFS library to users.
Currently, HIPPEROS uses a FATFS library directly, the aim is to introduce an indirection in order to 
avoid concurrent access that might occur.

\subsection{Task division}
The objective was subdivided into subtasks in order to reach gradually some objectives. 
This can give a better view of the progress made and that is quite in the Agile spirit.\\

The current objective is to implement a prototype of the service executing correctly the 
task to open and read a file. 

Once the goal reach, a list of improvements are described in the "Confluence" development tool, 
as "Integrate the FATFS client library interface into newlib", for instance.

\section{Flow description}
\subsection{First week 31/07 to 04/08}
The first days were dedicated to the discovery of the working environment and tools. 
HIPPEROS uses to work with a certain methodology near of the AGILE method but the pair-programing. 
Indeed, programmers are only one on each computer. So each of them have to chose their 
own IDE. The first day was therefore devoted to the installation of a workstation under GNU/Linux with 
all needed IDE, and different softwares.
Then, I was asked to compile and run a number of tests in order to understand the operation.
As the aim of the internship is to develop a prototype, that is a an indispensable step.\\

On Wednesday, there was a meeting to clearly explain the project and the different stages. 
The task to develop first a prototype of exchange has been stated this day.
On Friday, the first step to use a client/server call to obtain the read operation was almost reached.\\

During the second week, I set up tests on basic functions in order to control the development properly. 
Then, I implemented the function read.\\

I encountered a number of problems that required an outward look. 
At the end of the week, on Friday, a Pull Request on Github was created so that I could send my first job despite a persistent bug.\\

To date, the code I have produced has received an initial review. 
The next objective is to correct the things that have been reported.

\end{document}
