%-------------------------------------------------------------------------------%	PACKAGES AND OTHER DOCUMENT CONFIGURATIONS
%-------------------------------------------------------------------------------
\documentclass[12pt]{article}
%\usepackage[francais]{babel}
\usepackage[T1]{fontenc}
\usepackage[utf8]{inputenc}
\usepackage[top=3cm, bottom=3cm, left=3cm, right=3cm]{geometry}
\usepackage{fancyhdr}
\usepackage{lastpage}
\usepackage{graphicx}
\usepackage[colorinlistoftodos]{todonotes}
\usepackage{amsmath}
\usepackage{color}
\usepackage{array}
\definecolor{light-gray}{gray}{0.95}
\usepackage{xcolor}
\usepackage{listings}
\setlength{\parindent}{0pt}

\usepackage{path}
\usepackage[bookmarks,bookmarksnumbered,colorlinks,hyperindex,pagebackref,urlcolor=blue,final]{hyperref}

\definecolor{email}{rgb}{0,.5,0}

\definecolor{codegreen}{rgb}{0,0.6,0}
\definecolor{codegray}{rgb}{0.5,0.5,0.5}
\definecolor{codepurple}{rgb}{0.58,0,0.82}
\definecolor{backcolour}{rgb}{0.95,0.95,0.92}


\begin{document}
\begin{titlepage}

\def\EMAILArabella{Arabella.Brayer@ulb.ac.be}

\newcommand{\imagepath}{../img}
\newcommand{\university}{Université Libre de Bruxelles}
\newcommand{\faculty}{Département d'informatique}
\newcommand{\course}{TRAN-F-501 - Internship}
\newcommand{\authors}{\bfseries{Arabella \textsc{Brayer}} -- \emailArabella}
\newcommand{\supervisor}{Paul \textsc{Rodriguez}}
\newcommand{\HRule}{\rule{\linewidth}{0.5mm}} % Defines a new command for the horizontal lines, change thickness here
\newcommand{\emailArabella}{\href{mailto:\EMAILArabella?subject=[TRAN-F-501 - Internship]}{\color{\EMAILArabella}\path|\EMAILArabella|}}
\center % Center everything on the page
 
%-------------------------------------------------------------------------------
%	HEADING SECTIONS
%-------------------------------------------------------------------------------

\textsc{\LARGE \university}\\[1.5cm] % Name of your university/college
\textsc{\Large \faculty}\\[0.5cm] % Major heading such as course name
\textsc{\large \course}\\[0.5cm] % Minor heading such as course title

%-------------------------------------------------------------------------------%	TITLE SECTION
%-------------------------------------------------------------------------------\HRule \\[0.4cm]
{ \huge \bfseries Implementation of a file system service on the HIPPEROS micro-kernel}\\[0.4cm] % Title of your document
\HRule \\[1.5cm]
 
%-------------------------------------------------------------------------------%	AUTHOR SECTION
%-------------------------------------------------------------------------------
\begin{minipage}{0.4\textwidth}
\begin{flushleft} \large
\emph{Authors:}\\
Arabella \textsc{Brayer} -- \href{mailto:Arabella.Brayer@ulb.ac.be?subject=[TRAN-F-501 - Internship]}{email}
\end{flushleft}
\end{minipage}
~
\begin{minipage}{0.4\textwidth}
\begin{flushright} \large
\emph{Supervised by:} \\
\supervisor % Supervisor's Name
\end{flushright}
\end{minipage}\\[2cm]


%----------------------------------------------------------------------------------------
%	DATE SECTION
%----------------------------------------------------------------------------------------

{\large October, 23, 2017 }\\[2cm] % Date, change the \today to a set date if you want to be precise


%----------------------------------------------------------------------------------------
%	LOGO SECTION
%----------------------------------------------------------------------------------------

\includegraphics{\imagepath/logoULB.png}\\[1cm] 
 
%----------------------------------------------------------------------------------------

\vfill % Fill the rest of the page with whitespace

\end{titlepage}

\newpage


\pagestyle{fancy}
\fancyhf{}
\setlength\headheight{15pt}
\fancyhead[L]{Bi-weekly report}
\fancyhead[R]{}
\fancyfoot[L]{October, 23, 2017}
\fancyfoot[R]{\thepage}

\newpage

%\tableofcontents
%-----------------------------------------------------------
%\newpage
\section{Presentation}
\subsection{Recall of the subject}
The aim of the internship is to develop and integrate a new file system 
service for the real-time operating system \textbf{HIPPEROS}, 
that would provide the functionality of the libC library to users.
Currently, \textbf{HIPPEROS} uses a FATFS library directly, the aim is to introduce an indirection in order to 
avoid concurrent access that might occur.

\subsection{Current task}
This report is the last one. In the previous report, it was told that the 
remaining task was to integrate the service and library into \textbf{Newlib}.
Note that \textbf{Newlib} is a common library used to re-implement c 
libraries into alternative OS.

Concretely, it means that a \textbf{HIPPEROS} user should be able to use 
the file system including <stdio.h> header. The new interface has to be :\\
\textit{fopen}, \textit{fread}, \textit{fwrite}, \textit{fclose}.
Finally, the aim is to validate a last Pull Request with these changes.

\section{Flow description}
\subsection{First week 09/10 to 13/10}
The start of the first week was dedicated to the end of migration of the 
library. The lib has been moved into hapi, the "Hipperos" API.
The service has been mived into the service path. 
A first job of documentation has been done in order to understand how to 
port \textbf{Newlib}. 

Then, a first Pull Request has been opened to validate these first changes.

In the second part of the week, the work was to found difficulties by trying to 
port \textbf{Newlib}, step by step. First difficulties were discovered :\\
As \textbf{Newlib} makes a pre-treatment before to come in our implementation, 
then the buffer mode is automatically set, and the malloc function is called. 
There is no malloc support in \textbf{HIPPEROS}, so that is impossible. 
My advisor helped me to found solutions to disable the buffer mode.

\subsection{Second week 16/10 to 20/10}
It has been decided that my work would be cut in parts to reach goals 
step by step, because the whole aim was heavy to implement in one time.
Then the first step (in the beginning of the week) was to add a list 
of opened files in the service, then to port \textbf{Newlib}.
Some of difficulties has been found, like this one :\\
\textbf{Newlib} manages a list of opened files. When the task quit, it will 
close the opened files. But in statically linked, when a task quit, 
all the files are closed. That can lead to concurrency problems because
if two tasks are reading the same task and one is finished, the second 
task cannot read the file anymore. It is mandatory to call the \_Exit() 
function to quit. 

Finally, when this job was made, we had to rebase my 
branch to be synchronized with the current version of \textbf{HIPPEROS}.

\subsection{Conclusion}
The main goal of the second part of the internship has been reached. 
The library is integrated in \textbf{Newlib}. 

\end{document}
