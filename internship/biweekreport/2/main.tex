%-------------------------------------------------------------------------------%	PACKAGES AND OTHER DOCUMENT CONFIGURATIONS
%-------------------------------------------------------------------------------
\documentclass[12pt]{article}
%\usepackage[francais]{babel}
\usepackage[T1]{fontenc}
\usepackage[utf8]{inputenc}
\usepackage[top=3cm, bottom=3cm, left=3cm, right=3cm]{geometry}
\usepackage{fancyhdr}
\usepackage{lastpage}
\usepackage{graphicx}
\usepackage[colorinlistoftodos]{todonotes}
\usepackage{amsmath}
\usepackage{color}
\usepackage{array}
\definecolor{light-gray}{gray}{0.95}
\usepackage{xcolor}
\usepackage{listings}
\setlength{\parindent}{0pt}

\usepackage{path}
\usepackage[bookmarks,bookmarksnumbered,colorlinks,hyperindex,pagebackref,urlcolor=blue,final]{hyperref}

\definecolor{email}{rgb}{0,.5,0}

\definecolor{codegreen}{rgb}{0,0.6,0}
\definecolor{codegray}{rgb}{0.5,0.5,0.5}
\definecolor{codepurple}{rgb}{0.58,0,0.82}
\definecolor{backcolour}{rgb}{0.95,0.95,0.92}


\begin{document}
\begin{titlepage}

\def\EMAILArabella{Arabella.Brayer@ulb.ac.be}

\newcommand{\imagepath}{../img}
\newcommand{\university}{Université Libre de Bruxelles}
\newcommand{\faculty}{Département d'informatique}
\newcommand{\course}{TRAN-F-501 - Internship}
\newcommand{\authors}{\bfseries{Arabella \textsc{Brayer}} -- \emailArabella}
\newcommand{\supervisor}{Paul \textsc{Rodriguez}}
\newcommand{\HRule}{\rule{\linewidth}{0.5mm}} % Defines a new command for the horizontal lines, change thickness here
\newcommand{\emailArabella}{\href{mailto:\EMAILArabella?subject=[TRAN-F-501 - Internship]}{\color{\EMAILArabella}\path|\EMAILArabella|}}
\center % Center everything on the page
 
%-------------------------------------------------------------------------------
%	HEADING SECTIONS
%-------------------------------------------------------------------------------

\textsc{\LARGE \university}\\[1.5cm] % Name of your university/college
\textsc{\Large \faculty}\\[0.5cm] % Major heading such as course name
\textsc{\large \course}\\[0.5cm] % Minor heading such as course title

%-------------------------------------------------------------------------------%	TITLE SECTION
%-------------------------------------------------------------------------------\HRule \\[0.4cm]
{ \huge \bfseries Implementation of a file system service on the HIPPEROS micro-kernel}\\[0.4cm] % Title of your document
\HRule \\[1.5cm]
 
%-------------------------------------------------------------------------------%	AUTHOR SECTION
%-------------------------------------------------------------------------------
\begin{minipage}{0.4\textwidth}
\begin{flushleft} \large
\emph{Authors:}\\
Arabella \textsc{Brayer} -- \href{mailto:Arabella.Brayer@ulb.ac.be?subject=[TRAN-F-501 - Internship]}{email}
\end{flushleft}
\end{minipage}
~
\begin{minipage}{0.4\textwidth}
\begin{flushright} \large
\emph{Supervised by:} \\
\supervisor % Supervisor's Name
\end{flushright}
\end{minipage}\\[2cm]


%----------------------------------------------------------------------------------------
%	DATE SECTION
%----------------------------------------------------------------------------------------

{\large August, 26, 2017 }\\[2cm] % Date, change the \today to a set date if you want to be precise


%----------------------------------------------------------------------------------------
%	LOGO SECTION
%----------------------------------------------------------------------------------------

\includegraphics{\imagepath/logoULB.png}\\[1cm] 
 
%----------------------------------------------------------------------------------------

\vfill % Fill the rest of the page with whitespace

\end{titlepage}

\newpage


\pagestyle{fancy}
\fancyhf{}
\setlength\headheight{15pt}
\fancyhead[L]{Bi-weekly report}
\fancyhead[R]{}
\fancyfoot[L]{\today}
\fancyfoot[R]{\thepage}

\newpage

%\tableofcontents
%-----------------------------------------------------------
%\newpage
\section{Presentation}
\subsection{Recall of the subject}
The aim of the internship is to develop and integrate a new service for the real-time operating system HIPPEROS, 
that would provide the functionality of the FATFS library to users.
Currently, HIPPEROS uses a FATFS library directly, the aim is to introduce an indirection in order to 
avoid concurrent access that might occur.

\subsection{Current task}
The current task is still to implement a prototype of the service 
executing correctly the call to open and read file using the service. 
A task must be able to open several different files, in read mode.
Then, the task must be able to open a file, even if it is opened 
by another task in the system, read part by part without offset problem.

As a reminder, two weeks ago, I proposed my first prototype, but 
bugs were still there. My code was having its first review.

\section{Flow description}
\subsection{First week 13/08 to 17/08}
At the start of the week, the work has been to make corrections 
in line with the remarks made in the first review. 
However, a persistent bug continued to show up intermittently. 
By conquest, in the course of this week I was shown how to debug effectively.

A series of new remarks arrived on Wednesday, so 
I was busy to fix style-code and compliance.

The persistent bug was found at the end of this first week and was I guess a 
classical mistake for an intern with the C code. 
To fix this bug, I had to ask for help, which allowed me to learn how to 
better configure my IDE and use it professionally. 
As a personal remark, I think this error taught me a lot.

\subsection{Second week 20/08 to 25/08}
For the start of this week, I have been seek during one day.
From the second day, I undertook to complete the review by doing 
a lot of refactoring on the code. Then, I was shown how to properly 
maintain git deposits. 
As a personal remark, I think that these details can not be learned 
when developing individual projects, and difficulty in developing projects 
for the University, although in group.

Wednesday, I learned how to automate a test in order to apply this method on 
the prototype. The test used for HIPPEROS is not usually used, it is a custom, 
I am at this step still implementing these tests.
A first script was produced as a test, but this test is not yet complete.

\section{Next goal}
During the next week, my internship supervisor is not by HIPPEROS, so we 
had a meeting to organize the work. 
The prototype is expected to pass the review soon, then the library will be 
integrated at the kernel as a service. Then, the next step is to add 
the write operation.

\end{document}
