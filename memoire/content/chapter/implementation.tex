\section{Introduction}

L'implémentation d'un algorithme d'ordonnanceur décrit dans un papier théorique 
est une étape importante. 
Ce moyen va en effet permettre de valider la faisabilité d'un modèle 
mathématique.
Ce n'est pas parce qu'un algorithme est prouvé mathématiquement 
qu'il est implémentable dans un \textbf{RTOS}, aussi cette étape va permettre de le vérifier.
Il y a également un autre avantage à implémenter un ordonnanceur décrit uniquement 
de façon théorique et uniquement simulé
Par exemple, il n'est pas rare de voir des papiers évoquer des proportions 
dans leurs calculs, notamment pour le calcul de l'utilisation \todo{renvoi}.
Or, chacun sait que les ordinateurs ne peuvent 
pas représenter des fractions avec une précision infinie. 
Dans d'autres cas, on se figurera un appel à l'ordonnanceur à des étapes 
précises, mais en pratique, cela peut se passer de façon différente, 
l'ordonnanceur étant finalement une tâche également, et devant recevoir 
lui-même la main afin de pouvoir être exécuté. Cela dépendra donc des 
différences entre un algorithme théorique et l'informatique discrète, de 
façon générale, mais également en fonction du RTOS choisi pour l'implémentation.





\section{Description}
	\subsection{Du papier au script (généralité)}
	La première étape pour entamer ce travail est de choisir un algorithme parmi ceux donnés dans l'état de l'art.
	Cette étape du choix peut être cruciale pour la suite car il faut que l'algorithme soit 
	intéressant \todo{décire intéressant},
	optimal \todo{rappel optimal ou renvoi}
	applicable en pratique \todo{expliquer ça}.
	Initialement, notre choix portait sur QPS, un 
	successeur de RUN, décrit dans l'état de l'art. Cependant, les détails des papiers n'ont pas donné 
	assez confiance quant à la possibilité d'implémentation et nous avons préféré nous tourner vers UEDF, finalement.
	
	
	Au moment où l'on choisit un algorithme, on ne sait pas encore si sa réalisation va être faisable, 
	ne serait-ce que parce que la description donnée est mathématique, donne des démonstrations, 
	basées sur des valeurs sans se soucier de comment on les a obtenues.
	
	En pratique, le \textbf{WCET} -- on en reparlera -- est une valeur on ne peut plus théorique, par exemple, 
	et on ne peut pas facilement la déterminer. 
	Aussi, le déroulement séquentiel de l'ordonnanceur nécessite que les autres tâches cessent d'être exécutées 
	à ce moment-là, ce qui ne correspond pas forcément à la réalité.
	On imagine aussi qu'à chaque événement, l'ordonnanceur est immédiatement appelé, 
	en pratique, cela n'est pas forcément le cas, en tout cas pas pour tous les 
	\textbf{RTOS}. Ce n'est pas le cas d'\textbf{HIPPEROS}, par exemple.

	Les objectifs d'un tel travail ont été énoncés plus tôt. 
	Pour résumer, on souhaite faciliter le passage de la connaissance théorique vers la pratique.
	Cela consiste à essayer d'implémenter un ordonnanceur décrit sur le papier, voir s'il est 
	possible de l'implémenter, décrire ce qui facilite la mise en œuvre ou la complique. 
	
	
	\subsection{UEDF : pas d'implémentation connue}
	UEDF n'a à ce jour pas d'implémentation connue au sein d'un RTOS. HIPPEROS est le seul OS à en 
	avoir une implémentation à ce jour. Des simulations existent, mais comme on en reparlera, une simulation diffère 
	en bien des points d'un véritable OS. C'est une des difficultés de ce travail, qui est de ne pas pouvoir 
	s'appuyer sur d'autres implémentations. Par ailleurs, il n'y a pas non plus de jeu de données que l'on 
	pourrait comparer. On reparlera de ce point. 
	
	Toutefois, l'algorithme UEDF est décrit très précisément dans le papier de G. Nelissen. La thèse y énonce 
	des démonstration, mais également, dans un déroulement logique, des cas d'utilisation avec 
	plusieurs types de RTOS. 
	
	Par exemple : un RTOS à temps discret, ou continu. 
	Évidemment, tous les ordonnanceurs implémentés sont en temps discret, au final, car on travaille avec 
	un ordinateur, qui compte en binaire. Toutefois, l'implémentation de UEDF dépend en fonction du type 
	d'ordonnanceur, et certains, \todo{aller chercher la def dans le papier} sont définis comme continus 
	contre discret. Nous avons considéré qu'HIPPEROS était à temps continu, et donc avons procédé à une implémentation incrémentielle, allant du plus simple vers le plus complexe, comme dans le papier.
	
	
	Dans son papier encore, G. Nelissen explique au fur et à mesure comment éviter des calculs 
	redondants qui empêcheraient purement et simplement le fonctionnement d'UEDF en cas d'utilisation réel.
	\todo{dev allot computation} En effet, UEDF demande le calcul de plusieurs variables incrémentielles, 
	qui demandent de parcourir tout le set de tâches, pour tous les coeurs. Cela induit une complexité computationnelle 
	élevée, qui pourrait le rendre non pratiquable.
	
	Comment son implémentation est-elle rendue possible ? Pour chaque événement, un algorithme de calcul va être appelé.
	Cela rend l'algorithme très gourmand. La faisabilité d'un tel algorithme dépend donc du fait que l'on fasse ce calcul souvent ou non, et de s'il est possible de conserver les données de façon persistante. 
	
	Nous avons choisi de post poser ces questions, afin de nous lancer dans une implémentation, quitte à 
	ce qu'elle ne soit pas réalisable, dans un premier temps. Cela a mené à des tests unitaires qui 
	passaient, en partie, et à une très mauvaise efficacité. 
	
	
	\subsection{UEDF : description détaillée}
	
	
	
	\subsection{Essai/Erreurs (méthode d'implémentation)}
	La première implémentation faite a été réalisée sur une version d'HIPPEROS qui a grandement été revue par la suite. Par conséquent, mon implémentation a dû être entièrement revue. C'est aussi une des choses à prendre en
	 considération quand on fait une implémentation de ce genre, on travaille sur un outil en développement, 
	 dont les méthodes d'accès aux variables peuvent changer etc.
	 
	 Ma première implémentation était fort complexe, avec des datastructures simples pour me permettre d'obtenir
	 quelque chose de fonctionnel. Toutefois, l'implémentation ne fonctionnait pas réellement, 
	 mais seulement par unitTest.
	 
	 \subsubsection{UnitTests}
	 Pour développer UEDF, j'ai écrit des tests unitaires, puis implémenté l'algorithme. 
	 
	 Pourquoi les tests unitaires ?
	 
	 Ça permet de tester chacune des fonctionnalités isolément, en théorie, si les tests unitaires passent, 
	 il ne doit pas y avoir trop de surprise en passant au cas réel.
	 
	 On peut procéder de deux façons qui comportent chacune des avantages et inconvénients.
	 
	 \paragraph{Avantages} 
	
	\subsection{Choix d'implémentation (datastructures, système domaine)}
		\subsubsection{choix d'implémentation : data-structures}
		\subsubsection{Domaine système}
		
	\subsection{tests unitaires}
	Le sujet des tests unitaires est largement documenté. \todo{inclure sources} Différentes méthodes 
	proposent des approches afin de choisir pour quel type de couverture de tests on opte.
	Les méthodes Agile par exemple abordent systématiquement la question. Dans la méthode XP (Xtreme Programing),
	il est recommandé de faire écrire les tests par d'autres, et au préalable. Cela permet d'éviter les biais 
	psychologiques humains qui peuvent largement fausser les résultats. En effet, 
	la connaissance du code peut induire une façon de tester. Si je ne connais pas son fonctionnement, 
	je ne vais pas être tentée de le tester d'une manière plutôt que d'une autre, en évitant de mettre 
	le code en échec.
	
	\todo{see Test Driven Development}
	
	Pour l'implémentation d'UEDF, dans ce travail solitaire, il n'a pas été possible de mettre cela en place.
	J'ai plutôt opté pour une couverture large du code, et de façon itérative. 	
	mon implémentation, ce qui a rendu le développement lent et fastidieux mais m'a permis de me rendre compte des 
	erreurs assez vite sur les fonctions.
	
	
	\todo{décrire mon ordinateur utilisé pour les tests}
	L'ensemble des scripts et tests réalisés pour ce travail ont été faits sur un 
	Lenovo Z51, doté d'un processeur I5-5200 cadencé à 2.20GHz 4 cœurs. 
	Le système d'exploitation était GNU/Linux, Ubuntu 18.04.2 LTS. 
	
	Afin de procéder à la compilation croisée, j'ai utilisé les scripts mis à disposition par HIPPEROS, 
	qui permettent également de générer une machine virtuelle afin de vérifier l'exécution du code. 
	Cela a permis la compilation de l'ordonnanceur au sein d'HIPPEROS sans l'exécution.
	
	Cette étape se rapproche assez de la simulation dans le sens où on va ajouter des cas seulement 
	en fonction de ce qui a été implémenté et de ce que l'on souhaite tester. Cette partie-là n'apporte pas 
	à proprement parler de renseignements sur l'efficacité du code, mais sur la correction de l'ensemble de tâches.
	Concrètement, on vérifie qu'à chaque instant, l'ordonnanceur a réservé du temps sur l'ensemble des cœurs 
	pour chaque job. 
	
	\paragraph{Quels cas tester ?}
	Afin de s'assurer de la correction de l'algorithme, il faut vérifier si on arrive à le mettre en échec quand 
	il doit l'être, ou le contraire. Cela nécessite de trouver des jeux de données intéressants. 
	Typiquement, on pourra tester qu'en ajoutant des tâches de priorités différentes, 
	l'ordonnancement change. 
	
	Prenons Global-EDF. Cet algorithme, décrit précédemment, 
	
	
	\todo{vertical / horizontal}
	
	Malheureusement, il n'en existe pas à ma connaissance. Aussi une partie du travail a consisté 
	à construire des ensembles de tâches intéressants. Cela nécessite d'avoir une bonne 
	compréhension de comment l'algorithme est censé fonctionner, afin de voir ses faiblesses.
	
	Cette étape est difficile, pour plusieurs raisons :
	\begin{itemize}
		\item Trouver des ensembles de tâches
		\item Construire des tests sans être influencé par la connaissance que l'on a de son code
	\end{itemize}
	
	
	Comme 
	
	
	Pour procéder à cette partie, j'ai d'abord implémenté un invariant, reprenant celui donné dans la 
	thèse de G. Nelissen : 
	
	\todo{aller chercher l'invariant}
	
	Cet invariant -- comme son nom l'indique, doit être vrai tout le temps. Néanmoins, il ne permet pas 
	dans le cas de tests unitaires vraiment de vérifier s'il y a une erreur avec l'ordonnanceur, mais bien avec 
	la logique.
	
	C'est un test nécessaire, mais totalement insuffisant.
	
	D'autres tests peuvent vérifier 
	
	
	\todo{UEDF fait pas comme d'autres, il utilise le wcet, il schedule pas quand c'est pas possible}
	
		
	\subsubsection{tests sur board}	
	
	

