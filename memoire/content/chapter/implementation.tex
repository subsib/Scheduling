\section{Introduction}

Le travail d'implémentation a commencé au sortir d'un stage chez HIPPEROS. 
Cette étape a permis la connaissance nécessaire de l'OS afin de retoucher les fichiers, 
être capable d'ajouter un ordonnanceur, créer des tests d'intégration supplémentaires, etc.
Aussi d'apprendre les règles de codage au sein de l'entreprise. 

Ceci dans le but d'apporter un ordonnanceur supplémentaire à RTOS, mais également de la 
l'expérience scientifique.




\section{Description}
	\subsection{Du papier au script (généralité)}
	La première étape pour entamer ce travail est de choisir un algorithme parmi ceux donnés dans l'état de l'art.
	Cette étape du choix peut être cruciale pour la suite car il faut que l'algorithme soit 
	intéressant \todo{décire intéressant},
	optimal \todo{rappel optimal ou renvoi}
	applicable en pratique \todo{expliquer ça}.
	
	Au moment où l'on choisit un algorithme, on ne sait pas encore si sa réalisation va être faisable, 
	ne serait-ce que parce que la description donnée est mathématique, donne des démonstrations, 
	basées sur des valeurs sans se soucier de comment on les a obtenues.
	
	En pratique, le WCET -- on en reparlera -- est une valeur on ne peut plus théorique, par exemple, 
	et on ne peut pas facilement la déterminer. 
	Aussi, le déroulement séquentiel de l'ordonnanceur nécessite que les autres tâches cessent d'être exécutées 
	à ce moment-là, ce qui ne correspond pas forcément à la réalité.
	On imagine aussi qu'à chaque événement, l'ordonnanceur est immédiatement appelé, 
	en pratique, cela n'est pas forcément le cas, en tout cas pas pour tous les 
	\textbf{RTOS}. Ce n'est pas le cas d'HIPPEROS, par exemple.

	Les objectifs d'un tel travail ont été énoncés plus tôt. 
	Pour résumer, on souhaite faciliter le passage de la connaissance théorique vers la pratique.
	Cela consiste à essayer d'implémenter un ordonnanceur décrit sur le papier, voir s'il est 
	possible de l'implémenter, décrire ce qui facilite la mise en œuvre ou la complique. 
	
	
	\subsection{UEDF : pas d'implémentation connue}
	UEDF n'a à ce jour pas d'implémentation connue au sein d'un RTOS. HIPPEROS est le seul OS à en 
	avoir une implémentation à ce jour. Des simulations existent, mais comme on en reparlera, une simulation diffère 
	en bien des points d'un véritable OS. C'est une des difficultés de ce travail, qui est de ne pas pouvoir 
	s'appuyer sur d'autres implémentations. Par ailleurs, il n'y a pas non plus de jeu de données que l'on 
	pourrait comparer. On reparlera de ce point. Initialement, notre choix portait sur QPS, un 
	successeur de RUN, décrit dans l'état de l'art. Cependant, les détails des papiers n'ont pas donné 
	assez confiance quant à la possibilité d'implémentation et nous avons préféré nous tourner vers UEDF, finalement.
	
	
	L'algorithme UEDF est décrit très précisément dans le papier de G. Nelissen. La thèse y énonce 
	des démonstration, mais également, dans un déroulement logique, des cas d'utilisation avec 
	plusieurs types de RTOS. 
	
	Par exemple : un RTOS à temps discret, ou continu. 
	Évidemment, tous les ordonnanceurs implémentés sont en temps discret, au final, car on travaille avec 
	un ordinateur, qui compte en binaire. Toutefois, l'implémentation de UEDF dépend en fonction du type 
	d'ordonnanceur, et certains, \todo{aller chercher la def dans le papier} sont définis comme continus 
	contre discret. Nous avons considéré qu'HIPPEROS était à temps continu.
	
	Ensuite, dans son papier, G. Nelissen explique au fur et à mesure comment éviter des calculs 
	redondants qui empêcheraient purement et simplement le fonctionnement d'UEDF en cas d'utilisation réel.
	
	
	
	\subsection{Essai/erreurs}
	La première implémentation faite a été réalisée sur une version d'HIPPEROS qui a grandement été revue par la suite. Par conséquent, mon implémentation a dû être entièrement revue. C'est aussi une des choses à prendre en
	 considération quand on fait une implémentation de ce genre, on travaille sur un outil en développement, 
	 dont les méthodes d'accès aux variables peuvent changer etc.
	 
	 Ma première implémentation était fort complexe.
	 
	
	\subsection{Mes choix}
		\subsubsection{choix d'implémentation : data-structures}
		\subsubsection{Domaine système}
		
	\subsection{tests}
		
	\subsubsection{tests sur board}	

