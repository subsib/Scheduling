
\section{Résultats sur l'implémentation}
\label{resultats}

	\subsection{Implémentation possible}
	
	Une des questions auxquelles nous devions répondre était la possibilité d'implémentation d'\textbf{UEDF}, qui 
	n'avait bénéficié jusque là d'aucune implémentation sur un RTOS.
	Le travail effectué a d'ores et déjà montré qu'il était possible de l'implémenter, ce qui est un premier résultat attendu.
	Il est néanmoins important de préciser que cette implémentation est encore 
	perfectible, et nous donnerons des pistes d'amélioration dans le chapitre suivant. \todo{ref}\newline
	
\section{Rapport WCET vs temps d'exécution moyen}

	La première difficulté rencontrée lors de la mise en place de tests a été de déterminer la façon 
	de configurer les ensembles afin que ceux-ci soit ordonnançables sans provoquer de 
	dépassement de WCET ou d'échéances. La réponse n'a pas été simple à apporter, puisque ce nombre -- comme 
	évoqué dans le chapitre précédent -- \todo{ref} dépend de plusieurs facteurs, dont 
	l'algorithme d'ordonnancement lui-même. \newline
	
	D'après nos tests, la façon de déterminer les WCET dépend également du type d'ensemble, du nombre de tâches 
	ainsi que du nombre de cœurs utilisés, donc de l'\textit{utilisation} de l'ensemble.\newline
	
	
	\subsection{Utilisation 100 \%}
	
	Nous avons analysé plusieurs ensembles dont la somme des utilisations était de $100\%$, et avons fait varier 
	le nombre de tâches. Un ensemble est choisi s'il ne provoque pas de dépassement de WCET ou d'échéance durant une exécution 
	mais qu'en variant de quelques dizaines de millisecondes la durée moyenne d'exécution de la tâche, une de ces erreurs arrive.\newline
	
	
	
	

\section{Résultats quant à la faisabilité dans l'industrie}

	

\section{Résultats des tests unitaires et sur board}

	

\section{Comparaison avec G-EDF}

	