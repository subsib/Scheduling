\section{Vérification de l'implémentation : tests unitaires}


L'implémentation d'un algorithme, quel qu'il soit, est souvent entouré de tests unitaires. 
Il n'y a pas spécialement d'intérêt scientifique à développer un paragraphe sur 
notre façon de tester \textbf{UEDF} si ce n'est pour faire une remarque.\newline

Une des difficultés rencontrée lors de l'implémentation d'\textbf{UEDF} a été de s'assurer de la correction de l'algorithme. 
Aucune autre implémentation n'était accessible sur un autre \textbf{RTOS}, de fait, et donc il n'y avait pas de 
comparaison possible. En outre, nous trouvons bien un invariant dans la thèse de G. Nelissen, ce qui assure 
que l'algorithme a bien réservé du temps pour toutes les tâches présentes dans l'ensemble 
(ce qui doit être vrai pour tout système où $instant\_utilisation <= m$), mais ne garantit pas 
qu'il n'y a pas d'erreurs ensuite dans la répartition des sous-systèmes, etc. Pour cela, il faut encadrer 
l'exécution d'autres tests. \newline

Nous avons programmé un simulateur en \textit{Python} afin de pouvoir comparer les résultats 
obtenus dans les deux cas. Évidemment, procéder de la sorte ne garantit pas que l'exécution soit correcte
puisqu'on ne confronte pas sa compréhension de l'algorithme. \newline

Une solution à cela serait de proposer des exemples d'ordonnancements, présentant des situations intéressantes :
\begin{itemize}
	\setlength\itemsep{0.1em}
	\item Migrations
	\item Enchevêtrements particuliers
	\item Évolution des valeurs
\end{itemize}
Fournir des illustrations d'exécutions type ne nous paraît pas forcément accessoire, car l'algorithme n'est pas facile à comprendre, 
et cela peut entraîner des erreurs d'implémentation. On peut considérer que c'est la tâche de la personne qui implémente, 
c'est probable, mais si l'objectif des papiers est de permettre la mise en œuvre des algorithmes proposés, 
il peut être intéressant d'aider. 


\section{Qu'est-ce qui rend un algorithme implémentable ?}

\section{Que proposer dans un papier pour faciliter le passage à l'implémentation ?}

