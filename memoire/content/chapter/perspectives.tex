\label{perspectives}
Dans cette partie, nous revenons sur le travail effectué et proposons des voies d'amélioration.
Ces propositions portent sur plusieurs domaines :
\begin{itemize}
	\item Des améliorations concernant l'implémentation
	\item De nouveaux tests pour compléter nos premières observations
	\item plus largement sur la littérature scientifique, lorsqu'une équipe souhaite proposer un algorithme nouveau
\end{itemize}

\section{Poursuivre l'implémentation}

	L'état actuel de notre implémentation n'est pas totalement satisfaisante sur bien des points et 
	gagnerait fort à être continuée.
	
	\subsection{Inverser les cœurs}
		Les résultats nous ont montré que le cœur \textit{Master} avait une charge de travail plus basse que les 
		autres cœurs. Cette observation découle directement d'un artefact : nous avons volontairement augmenté les 
		écarts entre le WCET et le temps moyen d'exécution de certaines tâches afin de rendre \textbf{UEDF} capable 
		d'ordonnancer. En l'absence de cette manipulation, nous avons constaté de nombreux dépassements et n'arrivions pas 
		à trouver d'ensembles satisfaisants à étudier.\newline
		
		Il est tout à fait possible d'éviter cette manipulation, et la modification n'est pas des plus compliquées : 
		en effet, en inversant l'ordre des processeurs dans l'algorithme, il est possible de changer cette tendance. 
		Dans ce cas, on peut tout à fait imaginer que le dernier cœur serait naturellement le moins chargé, et que par 
		conséquent, aucune manipulation particulière ne soit nécessaire afin de rendre l'algorithme bien plus performant au 
		niveau de la répartition des charges de travail.\newline
		
		Un premier essai a été fait, et a montré que le changement méritait de l'attention pour éviter d'incorporer des 
		erreurs, néanmoins reste de complexité tout à fait abordable.
		

	\subsection{Structures de données}
		L'implémentation et ses structures ont été exposées plus haut dans document. 
		Il serait sans aucun doute efficace d'optimiser les diverses structures utilisées. \newline
		
		Une amélioration viendrait d'un moyen efficace de conserver le \textit{Heap} et de le maintenir 
		à jour durant l'exécution. En effet, actuellement, ce \textit{Heap} est reconstruit à chaque 
		relâchement de tâche. Il gagnerait à n'ajouter que les nouvelles tâches et à retirer celles qui sont terminées.
		Cette amélioration a été testée brièvement, et s'est révélée prometteuse.

	\subsection{Ordonnanceur virtuel}
	
		Dans la thèse de G. Nelissen \cite{nelissen_geoffrey_efficient_2013}, une amélioration est proposée et serait très intéressante 
		à ajouter à notre implémentation. En effet, afin de limiter le nombre de migrations, l'auteur propose de conserver une 
		copie des décisions \og{}virtuelles\fg{} et de vérifier au moment d'une migration si celle-ci peut être évitée, en 
		échangeant la liste \textit{Eligible} d'un cœur donné avec celle d'un autre cœur. Dans ce cas, le nombre de migrations devrait 
		réduire drastiquement, ce qui pourrait aider à diminuer de beaucoup les surcoûts.
		Le coût d'une telle opération en complexité est relativement élevé, toutefois, nous avons vu que le temps de calcul 
		est moins coûteux que les migrations, aussi ce serait une dépense justifiée et très efficace.

\section{Tests supplémentaires}

	\subsection{Compléter les résultats actuels}
	
	En premier lieu, beaucoup de résultats mériteraient d'être plus détaillés. Il y aurait encore d'autres résultats à produire, 
	comme l'analyse du temps d'exécution de $ComputeSchedule$, mais sur plus d'échantillons, comme un détail plus 
	fin de l'écart type (nous savons que le temps passé dans cette fonction dépend beaucoup du fait qu'il y a eu un relâchement 
	de tâche auparavant ou pas, mais n'avons pas en l'occasion de présenter ces résultats). En bref, il resterait d'autres résultats à 
	montrer et détailler depuis les expériences que nous avons menées. 

	\subsection{De nouveaux résultats dans de nouvelles recherches}
	
	Nous avons limité nos recherches et expériences à des ensembles de tâches de périodes harmoniques \hyperref[harmonique]{[\ref{harmonique}]}, 
	ce qui est justifié par plusieurs arguments, développés dans la partie \hyperref[methodo]{Méthodologie [\ref*{methodo}]}.
	Néanmoins, en apportant les quelques modifications exposées ici, nous pensons qu'il serait envisageable et intéressant 
	de produire des résultats pour des ensembles de tâches 
	de périodes non harmoniques, avec des rapports plus complexes entre elles. Nous attendrions de constater 
	une meilleure répartition de la charge de travail, et beaucoup moins de migrations, et ce résultat doit encore 
	être démontré par des expérimentations.

\section{La littérature scientifique}

	Dans le cas de \textbf{UEDF}, les documents disponibles montrent un effort de l'équipe afin de rendre compréhensible 
	l'algorithme et permettre sa mise en œuvre. Comme cela a été dit précédemment, cet effort n'est pas lisible dans tous les 
	documents que nous avons eu l'occasion de parcourir, ce qui rend l'implémentation compliquée. Le niveau de complexité 
	d'un algorithme en lui-même n'est pas forcément équivalent à la difficulté d'implémentation dans un environnement 
	déjà écrit, partagé avec d'autres ordonnanceurs. Aussi, si cette complexité est déjà très grande, nous pensons que les 
	chercheurs devraient tout mettre en œuvre afin de rendre le passage du papier à l'implémentation plus facile, en illustrant 
	avec des exemples les calculs, en donnant des conditions, des invariants, et pourquoi pas proposer des cas particuliers 
	auxquels il faut particulièrement être attentifs. \newline
	
