\label{perspectives}
Dans cette partie, nous revenons sur le travail effectué et proposons des voies d'amélioration.
Ces propositions portent sur plusieurs domaines :
\begin{itemize}
	\item l'implémentation
	\item d'autres tests
	\item plus largement sur la littérature scientifique sur la question
\end{itemize}

\section{Poursuivre l'implémentation}

	L'état actuel de notre implémentation n'est pas totalement satisfaisante sur bien des points et 
	gagnerait fort à être améliorée. 
	
	\subsection{Inverser les cœurs}
		Dans \textbf{HIPPEROS}, c'est le cœur \textit{Master} qui exécute l'algorithme d'ordnnancement 
		Il serait intéressant d'échanger les cœurs afin d'alléger la charge de ce premier cœur. 
		Nous sommes toutefois réservé sur les résultats à attendre de cela, 
		puisque comme l'ont montré les résultats, la charge \todo{blabla résultats}
	
	\subsection{Structures de données}
		L'implémentation et ses structures ont été exposées plus haut dans document. 
		Il serait sans aucun doute efficace d'optimiser les diverses structures utilisées, par exemple 
		la liste $Eligible$ ainsi que ses accès. \newline
		
		Une amélioration viendrait d'un moyen efficace de conserver le \textit{Heap} et de le maintenir 
		à jour durant l'exécution. En effet, actuellement, ce \textit{Heap} est reconstruit à chaque 
		relâchement de tâche. Il gagnerait à n'ajouter que les nouvelles tâches et à retirer celles qui sont terminées.

	\subsection{Ordonnanceur virtuel}
	
		Dans la thèse de G. Nelissen \cite{nelissen_geoffrey_efficient_2013}, une amélioration est proposée et serait très intéressante 
		à ajouter à notre implémentation. En effet, afin de limiter le nombre de migrations, l'auteur propose de conserver une 
		copie des décisions \og{}virtuelles\fg{} et de vérifier au moment d'une migration si celle-ci peut être évitée, en 
		échangeant la liste \textit{eligible} d'un cœur donné avec celle d'un autre cœur. Dans ce cas, le nombre de migrations devrait 
		réduire drastiquement, ce qui pourrait aider à diminuer de beaucoup les surcoûts.
		
		\todo{voir explication et donner formule ?! }


\section{Tests supplémentaires}

	Nous avons limité nos recherches et expériences à des ensembles de tâches de périodes harmoniques \hyperref[harmonique]{[\ref{harmonique} ]}.
	Ceci nous a permis de faire des expériences avec un résultat prédictible dans \todo{add} $\%$ des cas.
	Or il serait également intéressant de réussir à produire un résultat similaire mais pour des ensembles de tâches 
	de périodes non harmoniques. Cela semble plus difficile compte tenu des migrations que cela entraîne durant l'exécution, 
	mais avec les améliorations que nous proposons plus haut, cela pourrait être étudié également.

\section{Que proposer dans un papier pour faciliter le passage à l'implémentation ?}

