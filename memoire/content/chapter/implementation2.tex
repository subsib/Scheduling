\chapter{Implémentation}

	Une partie du travail consistait à évaluer si \textbf{UEDF} était implémentable. 
	La réponse est oui, car nous avons ici une implémentation fonctionnelle et fidèle à la description 
	théorique de l'algorithme. \newline
	
	Néanmoins, il nous semble utile d'exposer nos choix, afin de permettre à de nouveaux candidats 
	à l'exercice de profiter de nos essais/erreurs. Également, cela permet de comprendre les 
	résultats qui suivent, et de proposer des améliorations pour le futur.

	\subsubsection{Données temporelles}
	
		Dans \textbf{HIPPEROS}, il existe plusieurs ordonnanceurs déjà implémentés. 
		Ils ont leur fonctionnement propre, et de façon générale, sont assez différents d'\textbf{UEDF} par 
		rapport aux variables auxquelles ils accèdent. 
		UEDF a besoin de l'\textit{utilisation} d'une tâche dans certains de ses calculs. Or, l'\textit{utilisation} 
		est une fraction. Nous souhaitons éviter la représentation à virgule flottante 
		pour gérer ces fractions, et par conséquent, cela nécessite un décalage des valeurs.\newline
		 
		Se pose alors la question du décalage à appliquer afin d'occuper une place raisonnable en ayant 
		le moins de pertes possibles.  
		La réponse à apporter diverge en fonction de plusieurs facteurs :
		Appelons le décalage $SHIFT$.
		\begin{itemize}
			\setlength\itemsep{0.1em}
			\item Pour rappel, l'\textit{Utilisation} est définie de la sorte : $U(i) \defeq \frac{C_i}{T_i}$. 
			%\todo{def mots, ajouter utilisation, deadline, signalétique}
			En pratique, c'est le \textit{WCET} et la période de la tâche que l'on utilise pour calculer cette fraction : 
			$\frac{WCET}{T_i}$.
			HIPPEROS stocke les valeurs temporelles $wcet$ et $period$ en entiers non signés sur $64$ bits.
			
			\item Le décalage $SHIFT$ doit être choisi en fonction du rang des valeurs que l'on peut obtenir.
			Nous pouvons négliger les tâches de moins de $1 \%$ d'utilisation pour conserver un rang de $1$ à $100\%$.
			Concrètement, nous pourrons avoir les valeurs de $\frac{SHIFT}{100}$ à $SHIFT$.
			\item En théorie, le nombre de processeurs est libre. En pratique, 
			notre matériel ne dépassera pas les 4 cœurs, aussi aurons-nous une utilisation maximale de $400\%$.
			Quand bien même nous irions plus loin, nous ne dépasserions pas les 8 cœurs pour des raisons 
			matérielles, donc 800\% : $SHIFT \times 8$.
			\item En reprenant l'algorithme en section \ref{algouedf}, 
			l'utilisation sert à multiplier la différence entre deux échéances de travaux différents.
			Cette valeur est libre, et on peut l'imaginer grande dans certains cas. 
			Admettons qu'une tâche ait une échéant de $8.000.000 \mu s$, et une autre du double, comme cela 
			peut tout à fait se croiser dans un cas pratique, on aura une différence de $8.000.000$ à multiplier
			par cette proportion. 
			La valeur limite d'un entier de $64$ bits étant $18.446.744.073.709.551.615$, nous pouvons estimer que la marge 
			est grande et nous laisse libre de choisir un décalage grand.
		\end{itemize}
		Pour ces raisons, dans notre implémentation, nous avons estimé que décaler les valeurs de $1000$ était un bon compromis. 
		
	\subsection{Structures de données}
		Dans ses nombreux calculs, \textbf{UEDF} fait appel à un certain nombre de données concernant les tâches ou les travaux.
		Mais l'ordonnanceur a également besoin d'accéder à des variables qui lui sont 
		spécifiques à divers moments de l'exécution. 
		En outre, il doit également les travaux n'ayant pas atteint leur échéance, 
		car, entre deux relâchements de tâche, un travail $i$ peut se terminer ou être préempté, 
		il continue cependant de \og{}peser\fg{} dans l'ordonnancement au temps $t$ tant que $t < d_i(t)$.
		\newline
				
		\paragraph{Allot}
		La valeur $Allot$ est primordiale et doit être conservée durant l'exécution.
		Elle permet de décider si une tâche doit être exécutée ou non.
		Elle devra être actualisée à chaque événement.
		Ainsi, on doit conserver un tableau à deux dimensions $Allot$, qui contient des données 
		temporelles (entiers non signés de $64$ bits). 
		Cette structure est de taille $m \times n$ en théorie. 
		En pratique, dans HIPPEROS, $m$ et $n$ sont des valeurs prédéfinies 
		dans chaque version. Le nombre de tâches peut être à ce jour $32$, $128$ ou $1024$.
		Il ne nous a pas semblé utile de changer cela. Nos tests ont tous pour configuration 
		un nombre de tâche maximum de 32. 
		\newline
		
		\paragraph{Liste triée de tâches}
		Il est nécessaire de maintenir à jour un ensemble trié de tâches, pour le 
		calcul d'$Allot$. Pour cette structure, notre choix a beaucoup évolué au cours du temps. 
		La première idée était le besoin de parcourir cette liste dans l'ordre, plusieurs fois, 
		d'y ajouter et d'en retirer des éléments. A priori, pour ces raisons, le \textit{Heap} ne 
		correspond pas bien à nos besoins, aussi avons-nous utilisé une liste liée en 
		premier lieu. Néanmoins, les mauvaises performances de la liste liée 
		nous ont amené à reconsidérer ce choix, d'autant plus que les parcours de listes étaient nombreux. 
		À l'heure actuelle, notre implémentation construit un \textit{Heap} avant le recalcul d'$Allot$.
		Ce \textit{Heap} est copié chaque fois avant d'en faire le parcours, dans les diverses parties du code.
		En pratique, lorsque des travaux sont relâchés, nous parcourons à l'heure actuelle 3 fois ce \textit{Heap}, 
		chaque fois copié au préalable. 
		Ce choix a fait diminuer drastiquement le surcoût par rapport au choix de la liste liée, 
		néanmoins reste perfectible.
		%\todo{ajouter citation qui dit qu'il faut pas optimiser avant d'avoir bien implémenté, knuth}
		Par exemple, il est tout à fait imaginable de ne pas reconstruire le \textit{Heap} à chaque fois qu'il faut 
		calculer $Allot$, il faudrait maintenir une structure persistante, et mieux gérer les activations ou suppressions. 
		C'est une amélioration simple qui n'a pas encore été faite, et pourrait réduire légèrement les surcoûts.
		%\todo{ptete essayer avant la remise, ça pourrait se faire facilement je pense.} 
		\newline
		
		\paragraph{Liste d'états}
		Une première chose à laquelle il faut être attentif est de rendre possible les accès 
		aux données, par exemple en conservant des tableaux de références, afin de minimiser les temps 
		d'accès. Cela nécessite de faire un choix : on optimise le temps d'accès en conservant plus de données. 
		L'ordonnancement en cours est ainsi doublement référencé : 
		\begin{itemize}
			\setlength\itemsep{0.1em}
			\item $Job\_id \leftarrow coreToSchedule_{core}$ permet d'obtenir le travail en cours d'exécution sur le processeur $core \in m$
			\item $core \leftarrow scheduledToCore_{job\_id}$ permet d'obtenir le processeur qui exécute le travail $job\_id \in n$.
		\end{itemize}

		Nous devons également conserver les divers états d'un travail. En effet, rappelons que dans \textbf{UEDF}, 
		un travail est considéré comme \textit{actif} tant qu'il n'a pas rencontré son échéance -- et ce, 
		même s'il est \textit{terminé}. Nous conservons donc un tableau d'états permettant 
		d'accéder facilement à cette information.
		De même, nous devons différencier l'état \textit{bloqué} d'un travail \textit{terminé}, ce qui n'est pas le cas 
		dans les autres ordonnanceurs implémentés dans HIPPEROS. Par exemple, dans \textbf{Global-EDF}, si un travail 
		attend une ressource, elle sera purement et simplement retirée de la liste de tâches actives, et lorsqu'elle 
		sera réactivée, elle sera réintroduite dans cette liste. Ce comportement n'est pas possible dans \textbf{UEDF}, 
		qui \og{}réserver\fg{} du temps, et \og{}prévoit\fg{} un ordonnancement. Si le travail sort, et est simplement 
		réintroduit, cela provoquerait un nouveau calcul d'$Allot$, basé sur le $WCET$, et pas le temps restant d'exécution.
		
	\section{WCET, et Temps d'exécution}
		
		Les autres ordonnanceurs implémentés dans HIPPEROS n'utilisent pas les données concernant le temps 
		d'exécution de la tâche dans leur prise de décision. 
		Pour \textbf{UEDF}, nous avons besoin d'un accès efficace et correctement mis à jour à ces données. 
		Typiquement, le \textbf{Temps d'exécution restant}(\textbf{RET}) est utilisé pour le calcul et la mise à jour 
		d'\textbf{Allot}.\newline
		
		En pratique, \textbf{HIPPEROS} met à jour le temps exécuté d'un travail lors d'un événement qui le concerne. 
		Ainsi, si un travail est effectué, on note le moment où le travail a été dispatché.
		Au temps \textit{t}, si l'on veut savoir combien de temps a déjà été exécuté, 
		s'il n'a pas croisé d'événement, il faudra calculer ce temps dans \textbf{UEDF} 
		car \textbf{HIPPEROS} n'aura pas encore mis à jour cette donnée.\newline
		Si en terme de temps d'accès, cette requête n'est pas très compliquée, il n'en demeure pas moins 
		que le résultat sera un peu approximatif, puisque l'ordonnanceur n'arrête pas l'exécution des tâches pendant 
		son exécution. Cela ne change pas fondamentalement le calcul car cela change des valeurs très faibles, 
		mais cela constitue une adaptation de la théorie en pratique.\newline

	\section{Modèle VS réel}
	
		Dans une situation modélisée, l'exécution s'arrête, l'algorithme d'ordonnancement est effectué, 
		puis les travaux sont dispatchés et l'exécution reprend.
		Le temps est comme arrêté, le surcoût n'est pas comptabilisé, et surtout, les décisions se prennent 
		immédiatement, en ayant des conséquences extrêmement rapides.
		\newline
		
		\todo{vérifier si pas de redite. Je dis sans arrêt la même chose je crois.}
		L'ordonnanceur ne fait pas partie le l'ensemble des tâches du système. Cependant, il est exécuté 
		sur un des processeurs, et occupe une charge non nulle de temps. Outre que cela fausse les calculs 
		liés à la charge possible (concrètement, les 100\% d'utilisation libres sur le cœur qui exécute 
		les actions de l'ordonnanceur ne sont plus 100\% mais sont emputées du surcoût), 
		cela créé un décalage entre le temps calculé au moment $t$ et le moment réel de l'exécution.\newline

		
		
		\todo{expliquer dispatcher model :}
		le dispacther demande le schedule, 
		le schedule est calculé, et retourné,
		le dispatcher exécute.
		les événements sont créés : block, kill, activate... et provoquent kill ou activate, donc faut 
		savoir pourquoi on se retrouve dedans computeSchedule.

		Si un ou des nouveaux travaux : un booléen indique qu'il faut recalculer Allot. Puis, 
		on fait l'assignation ensuite.
		S'il n'y a pas de nouveaux travaux mais qu'il y a un ou des travaux terminés :
		pour chaque processeur libre, prendre une décision pour la suite.\newline
		
		C'est aussi simple que ça, mais cela n'est pas décrit dans le papier et représente une 
		adaptation du modèle.
	
		\todo{chart représentant le déroulement d'une exécution ??}
	
		
	\subsubsection{WCET}
		\todo{dev beaucoup plus}
		Le \textbf{WCET}, dans UEDF, est une donnée centrale. Elle est utilisée dans beaucoup de calculs, et 
		son importance est grande pour le résultat. Or, cette donnée n'est pas évidente à produire. \newline
		
		Notre travail nous amène à nous pencher sur ce sujet, qui dépasse largement la portée de ce travail. 
		Néanmoins, voici ce que l'on peut retenir pour nos besoins :\\
		Ce sujet est documenté dans la littérature scientifique. Il n'est pas évident de déterminer le 
		\textbf{WCET} pour toutes les tâches. Mettons qu'une tâche doive faire des lectures/écritures, 
		ce temps-là devra être considéré. Il est possible de déterminer le nombre d'instructions, 
		et donc un temps théorique en fonction de la machine utilisée pour l'exécuter, mais cela dépend 
		parfois de l'exécution. En effet, certaines opérations, en fonction des données, ne vont pas prendre 
		le même temps, or, ce que l'on cherche à déterminer et le pire des cas.\newline
	
		Pour les besoins de ce travail, nous avons simplifié la question, considérant des tâches 
		non dépendantes les unes des autres, avons fait un calcul à la main pour évaluer le 
		nombre d'opérations, et avons vérifié le temps d'exécution des tâches en exécutant un grand nombre 
		de fois les tâches et pris le pire temps comme valeur de \textbf{WCET}. Cela ne garantit en fait 
		pas que le \textbf{WCET} soit réellement le pire des cas, mais cela est suffisant pour nos besoins ici.
	
	\newpage
	\todo{Revoir, c'est DE LA MERDE}
	\subsubsection{Utilisation instantanée et interruptions}

	L'algorithme d'ordonnancement d'\textbf{UEDF} utilise comme variable l'utilisation instantanée du système. 
	Pour rappel, celle-ci est définie par la somme des utilisations des tâches actives à l'instant $t$, 
	tâche active signifiant tâche ayant relâché un travail qui n'a pas encore atteint son échéance. 
	On considère donc ici pouvoir comptabiliser tous les processus en exécution. Ceci est 
	en fait assez théorique et pose certains problèmes.
	
	Le problème de l'évaluation du \textbf{WCET} posé précédemment, il faut à ce stade rappeler comment 
	fonctionne un système d'exploitation.
	
	D'un point de vue théorique, si l'on estime voir l'ensemble des tâches dans un ensemble uniforme, 
	en pratique, ce n'est pas forcément le cas. Dans ce travail, une partie de la complexité est 
	donc éloignée. NUMA = non-memory-uniform-access
	
	Linux traite chaque processus comme un thread. Les threads ont un domaine. 
	https://lwn.net/Articles/80911/
	https://hal.archives-ouvertes.fr/hal-01295194/document  sur le scheduler linux
	\todo{développer cette partie et sourcer}
	

	L'ordonnanceur n'est pas compté dans les tâches, c'est un module du kernel. Par conséquent, il n'est pas comptabilisé 
	dans l'utilisation instantanée. Il fait partie de ce tas que l'on nomme \og{}overhead\fg{} et dans lequel 
	on met tout ce qui prend du temps sans être régulier. 
	Migrations, ordonnanceur, interruptions... Ces temps ne font pas partie du système, mais ils occupent
	pourtant du temps.
	
	Ce n'est pas la seule tâche qui n'est pas comptabilisée dans l'ensemble, puisqu'on peut définir des 
	interruptions côté utilisateur. 
	Concrètement, on imagine une sonde qui envoie très rarement des données. On ne réserve 
	pas de temps, on vérifie régulièrement qu'elle a des données à traiter, et si c'est le cas, on les traite.\\
	C'est un cas d'utilisation du system domain : garantir que les données seront traitées vite, dès qu'elles arrivent, 
	mais ne pas attribuer une tâche pour le faire régulièrement.
	Dans un système d'exploitation, l'ordonnanceur est chargé d'ordonnancer les tâches selon leurs priorités. 
	Pour ce faire, la norme \textbf{POSIX} définit des niveaux de priorité pour les threads. 
	\textbf{HIPPEROS }se base sur la même approche, et définit plusieurs niveaux également. C'est ainsi 
	qu'un thread peut être de scope system, et dans ce cas, sa priorité est haute, ou de priorité 
	moindre, et il sera ordonnancé par \textbf{UEDF}.\newline
	
	Tout ceci montre les limites d'un modèle où l'on charge l'utilisation des processeurs à 100\% avant de 
	passer au suivant. Certes, l'utilisation sera optimisée, dans le sens où l'on 
	pourra utiliser moins de processeurs, mais en revanche, on 
	répartir moins bien la somme de travail, et atteint très vite une somme critique pour les processeurs utilisés.
