
	Dans ce chapitre, nous développons et expliquons nos choix d'implémentation.	
	En effet, il nous semble utile d'exposer nos choix, afin de permettre à de nouveaux candidats 
	à l'exercice de profiter de nos essais/erreurs. Également, cela permet de comprendre les 
	résultats qui suivent, et de proposer des améliorations pour le futur.\newline

	\subsubsection{Données temporelles}
	
		Dans \textbf{HIPPEROS}, il existe plusieurs ordonnanceurs déjà implémentés. 
		Ils ont leur fonctionnement propre, et de façon générale, sont assez différents d'\textbf{UEDF} par 
		rapport aux variables auxquelles ils accèdent. 
		\textbf{UEDF} a besoin de l'\textit{utilisation} d'une tâche dans certains de ses calculs. Or, l'\textit{utilisation} 
		est une fraction. Nous souhaitons éviter la représentation à virgule flottante 
		pour gérer ces fractions, et par conséquent, cela nécessite un décalage des valeurs.\newline
		 
		Se pose alors la question du décalage à appliquer afin d'occuper une place raisonnable en ayant 
		le moins de pertes possibles.  
		La réponse à apporter diverge en fonction de plusieurs facteurs :
		appelons le décalage $SHIFT$.
		\begin{itemize}
			\setlength\itemsep{0.1em}
			\item Pour rappel, l'\textit{utilisation} est définie de la sorte : $U(i) \defeq \frac{C_i}{T_i}$. 
			%\todo{def mots, ajouter utilisation, deadline, signalétique}
			En pratique, c'est le \textit{WCET} et la période de la tâche que l'on utilise pour calculer cette fraction : 
			$\frac{WCET}{T_i}$.
			HIPPEROS stocke les valeurs temporelles $wcet$ et $period$ en entiers non signés sur $64$ bits.
			
			\item Le décalage $SHIFT$ doit être choisi en fonction du rang des valeurs que l'on peut obtenir.
			Nous pouvons négliger les tâches de moins de $1 \%$ d'utilisation pour conserver un rang de $1$ à $100\%$.
			Concrètement, nous pourrons avoir les valeurs de $\frac{SHIFT}{100}$ à $SHIFT$.
			\item En théorie, le nombre de processeurs est libre. En pratique, 
			notre matériel ne dépassera pas les 4 cœurs, aussi aurons-nous une utilisation maximale de $400\%$.
			Quand bien même nous irions plus loin, nous ne dépasserions pas les 8 cœurs pour des raisons 
			matérielles, donc 800\% : $SHIFT \times 8$.
			\item En reprenant l'algorithme en section \ref{algouedf}, 
			l'\textit{utilisation} sert à multiplier la différence entre deux échéances de travaux différents.
			Cette valeur est libre, et on peut l'imaginer grande dans certains cas. 
			Admettons qu'une tâche ait une échéance de $8.000.000 \mu s$, et une autre du double, comme cela 
			peut tout à fait se croiser dans un cas pratique, on aura une différence de $8.000.000$ à multiplier
			par cette proportion. 
			La valeur limite d'un entier de $64$ bits étant $2^{64} - 1$ ($18.446.744.073.709.551.615$), nous pouvons estimer que la marge 
			est grande et nous laisse libre de choisir un décalage grand.
		\end{itemize}
		Pour ces raisons, dans notre implémentation, nous avons estimé que décaler les valeurs de $1000$ était un bon compromis.
		Par exemple, pour une tâche qui serait définie comme suit : $\tau_1 = \{w:1; d=p : 3\}$, 
		l'\textit{utilisation} sera $\frac{1000}{5000} = 200$.
		Une \textit{utilisation} de $1\%$ est stockée dans l'implémentation d'\textbf{UEDF} comme une utilisation de 
		$10$, tandis qu'une de $100\%$ vaudra $1000$. Ce décalage sert dans tout l'\hyperref[algouedf]{algorithme décrit en section [\ref*{algouedf}]}.
		Aussi, toutes les variables seront décalées d'autant, comme l'index du processeur, afin d'avoir un calcul cohérent. 
		Toutefois, le nombre d'unités de temps $allot$ 
		n'est lui-même pas décalé, afin de contenir le bon nombre d'unités de temps à exécuter pour la suite.		
		
	\subsection{Structures de données}
		Dans ses nombreux calculs, \textbf{UEDF} fait appel à un certain nombre de données concernant les tâches ou les travaux.
		Mais l'ordonnanceur a également besoin d'accéder à des variables qui lui sont 
		spécifiques à divers moments de l'exécution. 
		En outre, il doit également conserver les travaux n'ayant pas atteint leur échéance, 
		car, entre deux relâchements de tâche, un travail $i$ peut se terminer ou être préempté, 
		il continue cependant de \og{}peser\fg{} dans l'ordonnancement au temps $t$ tant que $t < d_i(t)$.
		\newline
				
		\paragraph{Allot}
		Les valeurs du tableau multidimensionnel $Allot$ sont primordiales et doivent être conservées durant l'exécution.
		Elles permettent de décider si une tâche doit être exécutée ou non.
		Elles devront être actualisées à chaque événement.
		Ainsi, on doit conserver un tableau à deux dimensions $Allot$, qui contient des données 
		temporelles (entiers non signés de $64$ bits). 
		Cette structure est de taille $m \times n$ en théorie. 
		En pratique, dans \textbf{HIPPEROS}, $m$ et $n$ sont des valeurs prédéfinies 
		dans chaque version. Le nombre de tâches peut être à ce jour $32$, $128$ ou $1024$.
		Il ne nous a pas semblé utile de changer cela. Nos tests ont tous pour configuration 
		un nombre de tâche maximum de 32. 
		\newline
		
		L'algorithme de calcul du tableau $Allot$ (\hyperref[algouedf]{$Compute Allot$[\ref*{algouedf}]}) attribue les travaux
		aux processeurs par ordre ascendant d'index. Notre implémentation s'est adaptée à \textbf{HIPPEROS}, car c'est le 
		cœur $0$ le \textit{Master}, qui exécute l'algorithme d'ordonnancement. Par conséquent, 
		nous avons inversé l'ordre, pour que le cœur $0$ soit le dernier \og{}rempli\fg{} par l'algorithme -- 
		ce -- dans le but d'éviter que les surcoûts rendent des ensembles non faisables. Ainsi, 
		pour $4$ cœurs, une exécution jusqu'à $300\%$ d'\textit{utilisation} laissera le cœur $0$ 
		uniquement occupé à gérer l'ordonnancement.
		
		\paragraph{Liste triée de tâches}
		Il est nécessaire de maintenir à jour un ensemble trié de tâches, pour le 
		calcul d'$Allot$. Pour cette structure, notre choix a beaucoup évolué au cours du temps. 
		La première idée était le besoin de parcourir cette liste dans l'ordre, plusieurs fois, 
		d'y ajouter et d'en retirer des éléments. A priori, pour ces raisons, le \textit{Heap} ne 
		correspond pas bien à nos besoins, aussi avons-nous utilisé une liste liée en 
		premier lieu. Néanmoins, les mauvaises performances de la liste liée 
		nous ont amené à reconsidérer ce choix, d'autant plus que les parcours de listes étaient nombreux. 
		À l'heure actuelle, notre implémentation construit un \textit{Heap} avant le recalcul d'$Allot$.
		Ce \textit{Heap} est copié chaque fois avant d'en faire le parcours, dans les diverses parties du code.
		En pratique, lorsque des travaux sont relâchés, nous parcourons à l'heure actuelle 3 fois ce \textit{Heap}, 
		chaque fois copié au préalable. 
		Ce choix a fait diminuer drastiquement le surcoût par rapport au choix de la liste liée, 
		néanmoins reste perfectible.
		%\todo{ajouter citation qui dit qu'il faut pas optimiser avant d'avoir bien implémenté, knuth}
		Par exemple, il est tout à fait imaginable de ne pas reconstruire le \textit{Heap} à chaque fois qu'il faut 
		calculer $Allot$, il faudrait maintenir une structure persistante, et mieux gérer les activations ou suppressions. 
		C'est une amélioration simple qui n'a pas encore été faite, et pourrait réduire légèrement les surcoûts.
		%\todo{ptete essayer avant la remise, ça pourrait se faire facilement je pense.} 
		\newline
		
		\paragraph{Liste d'états}
		Une première chose à laquelle il faut être attentif est de rendre possible les accès 
		aux données, par exemple en conservant des tableaux de références, afin de minimiser les temps 
		d'accès. Cela nécessite de faire un choix : on optimise le temps d'accès en conservant plus de données. 
		L'ordonnancement en cours est ainsi doublement référencé : 
		\begin{itemize}
			\setlength\itemsep{0.1em}
			\item $Job\_id \leftarrow coreToSchedule_{core}$ permet d'obtenir le travail en cours d'exécution sur le processeur $core \in m$
			\item $core \leftarrow scheduledToCore_{job\_id}$ permet d'obtenir le processeur qui exécute le travail $job\_id \in n$.
		\end{itemize}

		Nous devons également conserver les divers états d'un travail. En effet, rappelons que dans \textbf{UEDF}, 
		un travail est considéré comme \textit{actif} tant qu'il n'a pas rencontré son échéance -- et ce, 
		même s'il est \textit{terminé}. Nous conservons donc un tableau d'états permettant 
		d'accéder facilement à cette information.
		De même, nous devons différencier l'état \textit{bloqué} d'un travail \textit{terminé}, ce qui n'est pas le cas 
		dans les autres ordonnanceurs implémentés dans \textbf{HIPPEROS}. Par exemple, dans \textbf{Global-EDF}, si un travail 
		attend une ressource, la tâche correspondante au travail bloqué sera purement et simplement retiré de la liste de tâches actives, 
		et lorsqu'elle sera réactivée, elle sera réintroduite dans cette liste. Ce comportement n'est pas possible dans \textbf{UEDF}, 
		qui \og{}réserve\fg{} du temps, et \og{}prévoit\fg{} un ordonnancement. Si le travail sort, et est simplement 
		réintroduit, cela provoquerait un nouveau calcul d'$Allot$, basé sur le $WCET$, et pas le temps restant d'exécution.
		
	\section{WCET, et Temps d'exécution}
		
		Les autres ordonnanceurs implémentés dans \textbf{HIPPEROS} n'utilisent pas les données concernant le temps 
		d'exécution de la tâche dans leur prise de décision. 
		Pour \textbf{UEDF}, nous avons besoin d'un accès efficace et correctement mis à jour à ces données. 
		Typiquement, le \textbf{Temps d'exécution restant}(\textbf{RET}) est utilisé pour le calcul et la mise à jour 
		d'\textbf{Allot}.\newline
		
		En pratique, \textbf{HIPPEROS} met à jour le temps exécuté d'un travail lors d'un événement qui le concerne. 
		Ainsi, si un travail est effectué, on note le moment où le travail a été dispatché.
		Au temps \textit{t}, si l'on veut savoir combien de temps a déjà été exécuté, 
		s'il n'a pas croisé d'événement, il faudra calculer ce temps dans \textbf{UEDF} 
		car \textbf{HIPPEROS} n'aura pas encore mis à jour cette donnée.\newline
		Si en terme de temps d'accès, cette requête n'est pas très compliquée, il n'en demeure pas moins 
		que le résultat sera un peu approximatif, puisque l'ordonnanceur n'arrête pas l'exécution des tâches pendant 
		son exécution. Cela ne change pas fondamentalement le calcul car cela change très faiblement les valeurs, 
		mais cela constitue une adaptation de la théorie en pratique.\newline

	\section{Modèle VS réel}
	
		Dans un modèle, l'exécution s'arrête, l'algorithme d'ordonnancement est effectué, 
		puis les travaux sont dispatchés et l'exécution reprend.
		Le surcoût peut être totalement ignoré.
		\newline
		
		L'ordonnanceur ne fait pas partie de l'ensemble des tâches du système. Cependant, il est exécuté 
		sur un des processeurs, et occupe une charge non nulle de temps. Outre que cela fausse les calculs 
		liés à la charge possible (concrètement, les 100\% d'utilisation libres sur le cœur qui exécute 
		les actions de l'ordonnanceur ne sont plus 100\% mais sont amputés du surcoût), 
		cela créé un décalage entre les valeurs calculées au moment $t$ et le moment réel de l'exécution.
		Il faudra adapter les WCET des tâches en fonction de cela, ce qui implique 
		une partie du travail qui consiste à évaluer le surcoûts, ainsi que de proposer une 
		façon de déterminer les WCET de tâches.\newline
		
	\section{Événements et calcul de l'ordonnancement}

		Dans HIPPEROS, voici une exécution typique :
		\begin{itemize}
			\setlength\itemsep{0.1em}
			\item Des tâches sont activées
			\item Le \textit{dispatcher} demande à l'ordonnanceur de calculer l'ordonnancement
			\item Le \textit{dispatcher} prend les décisions de l'ordonnanceur et effectue les changements 
			s'il y a lieu (dispatcher une tâche, en préempter une autre, etc.)
			\item Des événements viennent réveiller le \textit{dispatcher}, comme la fin, 
			la préemption, ou le relâchement d'un travail. Entre deux événements, l'exécution 
			de chacun des travaux sur les processeurs a lieu sans interruption.
		\end{itemize}
		
		Concrètement, c'est une fonction \textit{compute\_schedule} qui est appelée, sans 
		que l'on sache forcément quel événement a provoqué son appel. \newline
		
		Or, dans \textbf{UEDF}, il est nécessaire de différencier les moments où un travail a été relâché 
		d'un autre, où l'on ne devra pas recalculer $Allot$. Il suffit dès lors d'activer un 
		booléen indiquant qu'un nouveau travail a été relâché. Si ce booléen est vrai, 
		l'algorithme calcule totalement $Allot$, dans le cas contraire, il ne fait que mettre ses valeurs à 
		jour, avant de prendre dans les deux cas une décision.\newline
	
	\section{Domaine système}
	Dans le monde des systèmes embarqués, la norme \textbf{P}ortable \textbf{O}perating \textbf{S}ystem \textbf{I}nterface (\hyperref{http://www-sop.inria.fr/chir/personnel/arias/NArgimoge/Document/node19.html}{POSIX}{POSIX}{POSIX})	
	est une spécification des fonctions de base d'un système d'exploitation. Afin de permettre la portabilité des programmes, 
	cette norme garantit des propriétés quant à l'implémentation de certaines fonctions. \newline	
	
	L'ordonnanceur est chargé d'ordonnancer les tâches selon leurs priorités. 
	La norme \textbf{POSIX} définit des niveaux pour les threads : \textit{local}, et \textit{System}. 
	Mais en pratique, \textbf{Linux} n'utilise pas cela, tous les \textit{Threads} sont 
	de niveaux \textit{System}, et il ne gère pas de \textit{Process} comme ensemble de \textit{Threads}.\newline

	Une variante importante ressort ici : \textbf{HIPPEROS} gère des \textit{Process}, 
	qui sont des ensembles de \textit{Threads}. L'ordonnanceur gère des \textit{Process}, qui ont un ou plusieurs \textit{Threads}, 
	et les priorités pour chacun d'eux sont locales.
	C'est dans le but de gérer des exceptions et interruptions au niveau utilisateur qu'\textbf{HIPPEROS } a implémenté 
	des niveaux différents de domaines. S'il est du domaine \textit{system}, un \textit{Process }est de priorité supérieure, quoi que décide 
	l'ordonnanceur. Cela s'explique de cette manière :\newline

	Concrètement, on imagine une sonde qui envoie très rarement des données. On ne réserve 
	pas de temps d'exécution régulier pour effectuer cette écoute, 
	on vérifie simplement régulièrement qu'elle a des données à traiter, et si c'est le cas, on les traite.\newline
	
	C'est un cas d'utilisation du domaine \textit{System} : garantir que les données seront traitées vite, dès qu'elles arrivent, 
	sans attribuer une tâche pour le faire régulièrement.\newline
	
	Notre solution pour gérer cette situation avec \textbf{UEDF} est simplement de faire passer en priorité 
	la tâche du domaine \textit{System} s'il y en a une, sans la comptabiliser dans \textbf{UEDF}. 
	Cela pourrait invalider potentiellement un système, puisqu'on peut en théorie 
	exécuter une proportion de temps qui n'a pas été comptabilisée dans l'ordonnanceur. 
	Nous conseillons dans ce cas d'adapter d'autant le WCET des tâches.
