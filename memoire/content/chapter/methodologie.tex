\section{Choix de l'ordonnanceur}

	\subsection{Pourquoi UEDF}
	
	L'état de l'art nous a montré un vaste choix d'algorithmes intéressants pour différentes 
	raisons et n'étant pas encore implémentés. La première étape du travail consiste donc 
	à départager ces algorithmes pour arrêter notre choix sur l'un d'entre eux.
	
	Le choix le plus rationnel vis à vis des objectifs devrait se faire en fonction des 
	promesses théoriques de performance et de stabilité. 
	En effet, pour faire avancer la connaissance et permettre d'augmenter la confiance des utilisateurs 
	potentiels, il est judicieux de choisir un algorithme dont on attend au moins ceci :
	\begin{itemize}
		\item Un algorithme global qui minimise le nombre de migrations
		\item L'ordonnanceur est en ligne, optimal pour la classe périodique
		\item Il n'a pas bénéficié d'une implémentation sur un \textbf{RTOS}
		\item Il promet des performances intéressantes
	\end{itemize}
	
	Le premier choix a été \textbf{RUN}, puis finalement, son descendant, \textbf{QPS}. 
	Toutefois, les papiers disponibles étaient assez théoriques et nous n'avons pas réussi à 
	entrer en contact avec les créateurs de ces algorithmes. Aussi nos questions restaient sans réponses.
	Et ceci est peut-être une première remarque à faire à propos de la possibilité d'implémenter 
	un ordonnanceur : avoir un contact possible avec une personne à son origine, voire qui a déjà 
	réussi une implémentation, rend le travail bien plus simple, voire possible , en fonction des 
	papiers d'origine.\newline
	
	
	
	Finalement, c'est donc un argument quant à la communication (clarté, possibilité de poser des questions) 
	qui a fixé le choix. Par conséquent, c'est \textbf{UEDF} qui a été choisi, car il respectait 
	les promesses énoncées auparavant en plus d'être très bien documenté, et nous pouvions 
	poser nos questions directement à son créateur. Néanmoins, ses paramètres sont moins idéaux, 
	et nous avons revu nos attentes quant à l'efficacité à la baisse. Cela n'a en rien 
	modifié l'objet scientifique, à savoir le regard critique et la proposition d'améliorations 
	afin de stimuler et faciliter des implémentations futures.\newline
	
	
	
	Nous avons déjà présenté précédemment \textbf{UEDF}, de façon globale et succincte. 
	Dans cette partie, nous allons un peu plus en profondeur.
	
	\textbf{UEDF} est un ordonnanceur qui comporte plusieurs intérêts pour une implémentation.
	Tout d'abord, il est principalement \textit{en ligne}\todo{verif glossaire}, ce qui n'est pas le cas de 
	la plupart des ordonnanceurs utilisés dans l'industrie.
	Ensuite, il est global.\todo{include lien vers glossaire} Or, la plupart des ordonnanceurs 
	globaux connus et implémentés ne sont pas optimaux (pour la classe périodique), pour la
	raison exposée au préalable dans ce travail : il est nécessaire d'avoir de la clairvoyance, 
	c'est à dire une connaissance relative du futur.
	
	\textbf{UEDF} fait face à ce problème en ayant un traitement non pas vertical, mais horizontal
	de l'ordonnancement des tâches.

		\subsubsection{Global EDF}
		Afin de montrer la particularité d'\textbf{UEDF}, nous commençons par rappeler le fonctionnement 
		d'un ordonnanceur bien connu et que l'on peut considérer comme "vertical" : \textbf{Global EDF}. \todo{link état de l'art}
		
		L'algorithme est extrêmement simple. 
		À un moment \textit{t}, l'ordonnanceur prend sa décision de cette façon :
		si un processeur est libre, il se voit attribué le travail de priorité supérieure parmi 
		tous les travaux actifs. Cela permet d'avoir un algorithme extrêmement simple :\newline
		
		
		On conserve une structure de données 
		ordonnée, comme un \textbf{Heap} \todo{glossaire} qui contient tous les travaux
		devant être exécutés. À chaque fois que l'ordonnanceur doit prendre 
		une décision, il lui suffit de prendre le travail de priorité supérieure et 
		de l'exécuter sur le processeur libre.
		
		L'algorithme ne calcule rien à propos de l'avenir, sa décision au moment \textit{t}
		n'est prise qu'en considérant les jobs à exécuter et la liberté d'un 
		processeur. Il se peut qu'un autre ordonnancement plus efficace 
		réussisse à ordonnancer un système que \textbf{Global EDF} ne puisse pas résoudre...
		Cela rend tout de même l'ordonnanceur "efficace" en terme de calculs, 
		puisqu'il comporte un Heap, de complexité $O(n\log n)$, qui sera mis à jour 
		à chaque changement d'état du job (relâché -> inséré, 
		exécuté -> retiré du Heap).
		
		En revanche, ce fonctionnement ne permet pas à cet ordonnanceur d'atteindre 
		l'optimalité pour la classe de systèmes périodiques, comme on peut le voir 
		avec cet exemple :
		\todo{montrer un exemple qui rende Global EDF non efficace pour un job}

		\subsubsection{UEDF}
		UEDF -- quant à lui -- résout le problème de l'ordonnancement en effectuant un pré-calcul 
		que l'on pourrait comparer à un partitionnement. Ce calcul est toutefois en ligne et 
		effectué à chaque fois qu'une tâche relâche un job dans le système.
		
		En résumé, à chaque fois qu'un travail est activé, un calcul va permettre 
		de réserver des portions de temps pour toutes les tâches actives 
		actuellement dans le système. Au lieu de ne prendre en considération que 
		les priorités et les processeurs libres, \textbf{UEDF }vérifie donc à 
		chaque nouvelle libération de tâche que l'ordonnancement peut se faire. 
		En théorie, on devrait donc savoir immédiatement si un système est ordonnançable, 
		avant même de croiser un dépassement d'échéance. 
		
		Ainsi, la décision n'est pas réellement prise à ce moment, 
		mais on sait déjà qu'un travail peut-être effectué sur un 
		processeur, en garantissant le respect de l'échéance.
		
		La décision est ensuite prise à chaque événement de ce type :
		\begin{enumerate}
			\item Relâchement de travail
			\item Travail terminé
		\end{enumerate}

		En l'occurrence, à ce moment, il suffit de procéder comme pour \textbf{Global EDF} et de sélectionner le job le plus prioritaire, en respectant 
		une règle simple : effectuer le job de priorité supérieure, sauf s'il 
		est déjà en train d'être exécuté sur un autre processeur. Auquel cas, on ne 
		fait pas de migration, on prend simplement le travail suivant sur la liste.\newline
		
		On peut conclure par une explication imagée afin d'avoir une compréhension du fonctionnement d'\textbf{UEDF}. À chaque nouveau travail ajouté dans l'ensemble de travaux, 
		il fait un calcul afin de créer des sortes de partitionnements des travaux. 
		Ensuite, chacun de ces sous-systèmes applique EDF avec une règle ajoutée qui consiste 
		à vérifier si le travail de priorité supérieure n'est pas déjà en cours d'exécution.
		
		On peut dire qu'\textbf{UEDF} sur un processeur revient à appliquer \textbf{EDF}. Il 
		demeure très différent dès lors qu'il y a plusieurs processeurs.
		
		
		\todo{cette explication serait même incompréhensible pour quelqu'un connaissant bien UEDF... REFORMULER !! }
		\todo{illustration simple avec dessin}
		
		Nous appelons ce fonctionnement "horizontal" car au lieu de remplir 
		en priorité les processeurs "libres", UEDF va remplir les 
		processeurs par ordre. Ainsi, pour une utilisation inférieure à 100\%, 
		on n'aura besoin que d'un seul processeur. Pour 200, 2 processeurs, bref, 
		pour une utilisation de m * 100 \%, on utilisera m processeurs.
		
\section{HIPPEROS}
	\customhighlight{HIPPEROS} (High Performance Parallel Embedded Real-time Operating Systems)
	est un \customhighlight{RTOS} (Real-Time Operating System) développé depuis plusieurs années par une spinoff de l'ULB.
	Il bénéficie des connaissances apportées par le monde de la recherche dans 
	le domaine des systèmes critiques avec multic\oe{}urs. Une de ses particularités 
	est sa modularité, qui permet d'adapter ses possibilités en fonction du système 
	lors de la compilation de l'OS, ainsi peut-on différencier principalement 
	deux installations en fonction des particularités. 
	
	\customhighlight{HIPPEROS} est un candidat idéal pour l'implémentation d'un ordonnanceur 
	global. Il a cependant un fonctionnement propre qui pourra rendre l'implémentation 
	plus ou moins facile, et poser un certain nombre de problèmes. 
	En résumé, une nouvelle implémentation sur un OS différent 
	peut elle-aussi apporter à la connaissance générale des détails importants.
		
		

\section{Difficultés liées à l'implémentation}

	Dans cette partie, nous allons analyser les difficultés rencontrées lors de l'implémentation en elle-même, 
	en nous concentrant sur celles liées aux aspects théoriques d'origine. 

	\subsubsection{Accès aux données nécessaires}
	
		Dans \textbf{HIPPEROS}, il existe d'autres ordonnanceurs déjà implémentés. Il convenait donc 
		de s'adapter à l'implémentation existante. Cela signifie des accès parfois compliqués afin d'accéder 
		aux informations de la tâche.\newline
		
		\textbf{HIPPEROS} ordonne les tâches de cette façon : 
		\begin{itemize}
			\item Les données concernant les tâches sont accessibles via un objet \textit{task}. Ces 
			données sont statiques.
			\item Certaines données concernent une autre structure, appelée \textit{process}. L'accès 
			à cette structure ne se fait pas de la même façon. 
			\item Les données concernant le travail sont dans la structure \todo{je saisplus}.
		\end{itemize}
		Une première chose à laquelle il faut être attentif est de rendre possible les accès 
		à ces données, par exemple en conservant des tableaux de références, afin de minimiser les temps 
		d'accès. Cela nécessite de faire un choix : on optimise le temps d'accès en conservant plus de données.\newline
	
		Les autres ordonnanceurs n'utilisent pas vraiment les données concernant le temps 
		d'exécution de la tâche. Pour \textbf{UEDF}, nous avons besoin d'un accès efficace et 
		correctement mis à jour à ces données. Typiquement, le \textbf{RET} est utilisé en plein 
		milieu du calcul \textbf{Allot}, ce qui signifie que lors du calcul, il doit être possible 
		de récupérer ces données. Or, cela n'est pas tout à fait possible, ou du moins, pas idéalement. 
		En fait, \textbf{HIPPEROS} met à jour le temps exécuté d'une tâche lors d'un changement d'état de celle-ci. 
		Ainsi, si un travail est effectué, on note dans ses états le moment auquel il a été dispatché.
		Au temps \textit{t}, si l'on veut savoir combien de temps il a été effectué, s'il n'a pas changé 
		d'état, il faudra calculer ce temps dans \textbf{UEDF} car \textbf{HIPPEROS} n'aura pas encore mis à 
		jour cette donnée.\newline
		Si en terme de temps d'accès, cette requête n'est pas très compliquée, il n'en demeure pas moins 
		que le résultat sera un peu approximatif.
		
		
		Une des critiques que l'on peut adresser à l'algorithme \textbf{UEDF} est sa complexité. 
		Pour rappel, à chaque ajout de travail dans l'ensemble des tâches actives, il faut 
		parcourir la liste entièrement, et ce plusieurs fois, ce qui mène à 
		une complexité de \todo{chercher... $O(m \times n)$ ?}. 
		Certes, avec une astuce, proposée par G. Nelissen dans sa thèse, il est possible de réduire 
		la complexité du calcul, néanmoins pas sa fréquence. Il est donc primordial de réussir à 
		minimiser la complexité, et maximiser l'efficacité des structures de données utilisées par l'algorithme.\newline
		
		
		En théorie, lorsqu'on doit parcourir une liste ordonnée et supprimer des éléments régulièrement, 
		qui ne sont forcément la tête de la liste, le Heap est assez mal adapté. En premier lieu, 
		nous avons donc implémenté une liste liée. Ces listes sont connues pour avoir de mauvaises 
		performances, et elles impactaient énormément le temps de calcul d'\textbf{UEDF}. 
		Nous verrons que par la modification de cette structure, nous avons déjà réduit drastiquement 
		le temps d'exécution du calcul. Est-ce suffisant pour permettre à \textbf{UEDF} d'être utilisé dans la réalité est 
		une question à laquelle nous répondrons dans le prochain chapitre.
		
		
		
	
	\subsubsection{Complexité/persistance}
	\subsubsection{WCET}
	\subsubsection{S'arrête à chaque event}
	\subsubsection{Utilisation de 100\%, mais et les interruptions alors ?}

\section{La vérification de l'implémentation : tests unitaires simulés avec des sets créés}

\section{Tests globaux}

\section{Comparaison avec G-EDF}
