	

\section{La vérification de l'implémentation : tests unitaires simulés avec des sets créés}

	Une des difficultés rencontrée lors de l'implémentation d'\textbf{UEDF} a été de s'assurer de la correction de l'algorithme. 
	Aucune autre implémentation n'était accessible sur un autre \textbf{RTOS}, de fait, et donc il n'y avait pas de 
	comparaison possible. En outre, nous trouvons bien un invariant dans la thèse de G. Nelissen, ce qui assure 
	que l'algorithme a bien réservé du temps pour toutes les tâches présentes dans l'ensemble 
	(ce qui doit être vrai pour tout système où $instant\_utilisation <= m$), mais ne garantit pas 
	qu'il n'y a pas d'erreurs ensuite dans la répartition des sous-systèmes, etc. Pour cela, il faut encadrer 
	l'exécution d'autres tests. \newline

	Avant de nous lancer dans l'exécution réelle de l'algorithme, une grande partie du travail a consisté 
	à simuler l'exécution du programme à l'aide de tests unitaires. Ceux-ci ont été pensés de sorte à avoir 
	une couverture relativement élevée du code (50 \% environ) afin d'éviter les surprises lors du passage 
	

\section{Tests globaux}

\subsection{Génération de tâches}

\subsection{Surcoûts}

\section{Comparaison avec G-EDF}
