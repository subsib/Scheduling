\section{Enjeux et attente}

	Nous avons plusieurs attentes dans ce travail que vous souhaitons vérifier avec des tests.
	
	Nous voulons :
	\begin{enumerate}
		\item Vérifier qu'on a bien implémenté le bon algorithme
		\item Vérifier qu'il se comporte bien comme il faut...
		\item Regarder ses temps d'overheads sur des tests basiques
		\begin{enumerate}
			\item mesurer les WCET
			\item mesurer les temps d'exécution et retirer tout ce qui n'est pas de l'exécution = overhead
			\item voir si ça augmente bien avec le nombre de coeur, avec plus ou moins de tâches
		\end{enumerate}
		\item voir si avec des tâches de périodes harmoniques, on arrive à des trucs sympas
		\item mesurer le temps d'occupation des procs
		\item comparer pour tout ça avec Global-EDF
		\item trouver des trucs faisables avec UEDF, pas faisables avec global, et inversement
	\end{enumerate}
	
	
\section{Machine utilisée}
	HIPPEROS est développé pour pouvoir être installé sur un nombre limité de machines. 
	L'une d'elle est une SabreLite dont voici la description :
	
	\subsubsection{SabreLite truc}
	

\section{Génération d'ensembles de tâches}

	Nous n'avons pas utilisé de générateur totalement aléatoire de WCET et périodes, 
	car nous avons préféré tester pour des périodes harmoniques. Aussi 
	avons-nous opté pour des valeurs \og{}rondes\fg{} donc le nombre de combinaison est infini 
	mais pas infiniment intéressant.
	
	Nous avons testé des valeurs de plus en plus grandes sur des proportions comparables pour fournir 
	une méthode afin de fournir les WCET etc. pour cet ordonnanceur.
	
	Cette partie-là était chronophage.

\section{Détermination des WCET des tâches}
	\textbf{UEDF} est optimal, en \underline{théorie}. Revenons un instant sur ce point. 
	Cela signifie qu'en omettant tous les surcoûts, l'on est capable de charger les processeurs 
	à $100\%$ d'utilisation. 
	
	wcet = pas cet
	surcout
	machine

Pour déterminer les WCET, une série de tests a été effectués, sur des sets différents, 
avec des WCET plus ou moins pessimistes. 
On a mesuré les temps d'idle sur les processeurs, 
vérifié la faisabilité pendant plusieurs heures, 
dans le cas où le système passait, ou pas, nous avons conservé ces résultats.

ajouter plus tard :
Nous avons trouvé que le proc 1 gérait bcp d'overheads, et que par conséquent, il
lui faut des tâches avec WCET très pessimistes.

Nous avons testé avec 8 tâches également, et avons déterminé quelle 
type de famille de tâches pouvait fonctionner. Pour ce faire, nous avons utilisé un 
générateur de tâches.

En gros, une tâche effectue une boucle de calculs jusqu'à avoir atteint un temps donné, puis elle se ferme. 
Cela donne un ordre de grandeur de durée de tâche et permet d'éviter d'écrire sur la sortie, ce qui 
génère des temps longs difficilement mesurables.

		
\subsubsection{WCET}
Le \textbf{WCET}, dans UEDF, est une donnée centrale. Elle est utilisée dans beaucoup de calculs, et 
son importance est grande pour le résultat. Or, cette donnée n'est pas évidente à produire. \newline

Notre travail nous amène à nous pencher sur ce sujet, qui dépasse largement la portée de ce travail. 
Néanmoins, voici ce que l'on peut retenir pour nos besoins :\\
Ce sujet est documenté dans la littérature scientifique. Il n'est pas évident de déterminer le 
\textbf{WCET} pour toutes les tâches. Mettons qu'une tâche doive faire des lectures/écritures, 
ce temps-là devra être considéré. Il est possible de déterminer le nombre d'instructions, 
et donc un temps théorique en fonction de la machine utilisée pour l'exécuter, mais cela dépend 
parfois de l'exécution. En effet, certaines opérations, en fonction des données, ne vont pas prendre 
le même temps, or, ce que l'on cherche à déterminer et le pire des cas.\newline

Pour les besoins de ce travail, nous avons simplifié la question, considérant des tâches 
non dépendantes les unes des autres, avons fait un calcul à la main pour évaluer le 
nombre d'opérations, et avons vérifié le temps d'exécution des tâches en exécutant un grand nombre 
de fois les tâches et pris le pire temps comme valeur de \textbf{WCET}. Cela ne garantit en fait 
pas que le \textbf{WCET} soit réellement le pire des cas, mais cela est suffisant pour nos besoins ici.

\subsection{Génération de tâches}



\subsection{Surcoûts}

\section{Comparaison avec G-EDF}
