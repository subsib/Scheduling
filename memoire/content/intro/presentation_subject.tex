	
	Le secteur des appareils à équipement informatisé montre un essor important du 
	nombre de systèmes embarqués actuellement, et
	parmi eux, d'un nombre très important de systèmes temps réel qui 
	nécessitent des ordonnanceurs adaptés. 
	Ce champ de recherche a été largement nourri durant les vingt dernières années par de nombreuses publications scientifiques. \medskip
	
	Les systèmes embarqués n'ont pas toujours de grands besoins en efficacité, 
	ils doivent prioritairement être sûrs, et réactifs. 
	Toutefois, un système dont l'efficacité n'est pas maximale n'utilise pas 
	ses ressources de façon optimale, ce qui conduit à avoir recours à 
	plus de ressources que nécessaire. En effet, lorsqu'un système est 
	mal optimisé, il est sur-dimensionné, car plus de matériel est nécessaire 
	pour réaliser les mêmes tâches. Il est facilement imaginable que cela représente du 
	gaspillage d'espace (plus de volume occupé), 
	d'énergie électrique dédiée à l'alimentation des appareils surnuméraires, 
	de composants, ainsi que de matériaux de câblage.
	Or, la gestion énergétique 
	est un poste coûteux qui se doit d'être le plus réduit possible. 
	En l'occurrence, pour des installations équipées de beaucoup de systèmes embarqués, 
	comme les voitures ou les avions, cette dépense peut être substantielle. \medskip
	
	Par ailleurs, cela fait plusieurs années que la croissance de l'efficacité des 
	appareils n'est plus principalement due à la miniaturisation des composants. 
	Actuellement, les progrès engrangés portent sur la parallélisation, 
	ou encore, dans le domaine de l'architecture des composants eux-mêmes.
	Durant une cinquantaine d'années, dans le passé, les lois de \customhighlight{Moore}
	se sont vérifiées : des règles empiriques qui postulent que la fréquence des processeurs 
	double tous les dix-huit mois environ, à prix constant. 
	Il était alors encore possible de miniaturiser les composants, 
	ce qui faisait gagner en efficacité d'autant. 
	Cela n'est actuellement plus le cas, 
	les composants étant déjà largement miniaturisés, de nouvelles miniaturisations 
	n'entraîneraient plus de gain. La croissance doit emprunter d'autres voies. Parmi celles 
	empruntées, outre de nouvelles propositions technologiques, 
	il y a l'amélioration logicielle avec notamment la parallélisation.
	\medskip
	
	Le monde des systèmes embarqués en temps réel a bénéficié de ces améliorations, 
	et dispose également de processeurs multi-c\oe{}urs. Cependant, leur gestion n'est 
	à l'heure actuelle pas optimale, la plupart des systèmes sont  
	gérés soit comme des systèmes mono-processeurs, soit avec des algorithmes partitionnés \cite{paolillo_new_nodate}. 
	Dans le premier cas, les ressources ne sont pas utilisées de façon optimale, 
	et dans le second, des implémentations et tests ont montré empiriquement 
	que les algorithmes globaux pouvaient présenter des avantages intéressants, comme 
	une meilleure répartition de l'utilisation des processeurs \cite{baker_analysis_2005}. 
	\medskip
	
	Le décalage entre la connaissance scientifique et les implémentations réelles 
	peut s'expliquer en partie par le fait qu'il soit compliqué de gérer le partage des ressources, 
	et que cette complexité n'apporte pas suffisamment d'avantages à l'heure qu'il est.\medskip
	
	Le fait que l'industrie n'implémente pas à ce jour de solutions plus \og performantes \fg{} 
	pour les systèmes embarqués pose plusieurs problèmes :
	\begin{enumerate}
		\item Le matériel n'est pas exploité de façon optimale.
		\item Par conséquent, la consommation en énergie des solutions déployées n'est 
		pas optimale.
		À l'heure actuelle, dans un monde où l'on cherche à consommer le moins possible, 
		cela pose question. Mais au delà, cela signifie également plus de maintenance 
		sur ces appareils parfois sans source de renouvellement d'énergie.
		\item Certains systèmes embarqués nécessitent une très basse consommation car 
		ils sont difficiles d'accès ou non faciles à recharger.
		\item Sur de grosses installations, comme des voitures ou un avions, cela peut représenter 
		un coût important en terme d'occupation de l'espace, de câblage entre différents systèmes, 
		consommation d'énergie, poids, achat de composants.
	\end{enumerate}
	
	Toutes ces raisons poussent à s'intéresser à une implémentation réelle et réaliste 
	d'ordonnanceurs connus dans la littérature, mais moins dans la réalité.
	
	C'est dans ce contexte que s'inscrit ce travail d'implémentation d'un ordonnanceur en temps réel 
	global et multi-processeur, dont un des objectifs est d'améliorer la connaissance pratique 
	d'ordonnanceurs multi-processeurs globaux, d'en apercevoir les limites. 
	Cette implémentation pourrait amener plusieurs résultats intéressants : \medskip
	\begin{itemize}
		\item Une comparaison entre des résultats théoriques et l'effet de leur mise 
		en \oe{}uvre
		\item L'origine de ces différences
		\item Vérifier les avantages, inconvénients ou obstacles à la 
		commercialisation de telles solutions.
	\end{itemize}
	
	\section{Contexte et objectifs}
	
	\subsection{Ordonnanceurs globaux}
	Comme il sera expliqué plus tard dans ce document, il existe deux grandes familles d'ordonnanceurs 
	multi-processeur :\medskip
	\begin{itemize}
		\item Les ordonnanceurs partitionnés
		\item Les ordonnanceurs globaux
	\end{itemize}
	Si parmi ces deux familles d'ordonnanceurs, les systèmes mono-processeurs, voire 
	partitionnés ont le plus de succès auprès de l'industrie, 
	ce n'est pas la famille la plus efficace pour tout type de classe de tâches. 
	Les algorithmes globaux répartissent habituellement mieux l'utilisation des processeurs, 
	les migrations sont possibles et donc les temps de réponse sont généralement moins grands.
	\medskip
	
	
	En 1988, Hong and Leung \cite{hong_-line_1988} publient un article dans lequel est 
	démontré que :\\
	\textit{\og Il n'existe pas d'ordonnanceur multi-processeur en ligne optimal pour un système de tâches 
		avec plusieurs échéances distinctes\fg{}}.
	
	Toutefois, des publications ultérieures viennent compléter cette preuve, et démontrent 
	que des ordonnanceurs globaux optimaux existent, mais nécessitent de la clairvoyance.
	
	\subsection{HIPPEROS}
	\customhighlight{HIPPEROS} (High Performance Parallel Embedded Real-time Operating Systems)
	est un \customhighlight{RTOS} (Real-Time Operating System) développé depuis plusieurs années par une spinoff de l'ULB.
	Il bénéficie des connaissances apportées par le monde de la recherche dans 
	le domaine des systèmes critiques avec multic\oe{}urs. Une de ses particularités 
	est sa modularité, qui permet d'adapter ses possibilités en fonction du système 
	lors de la compilation de l'OS, ainsi peut-on différencier principalement 
	deux installations en fonction des particularités.
	
	\customhighlight{HIPPEROS} est un candidat idéal pour l'implémentation d'un ordonnanceur 
	global, mais une partie du travail consistera à tirer parti de ses particularités. 
	Par exemple, ce système d'exploitation gère les c\oe{}urs en leur attribuant des 
	niveaux différents. L'un est considéré comme \og maître \fg{} et les autres comme \og esclaves\fg{}. 
	Ceci peut apporter un comportement particulier, ce auquel il convient d'apporter 
	l'attention nécessaire. En résumé, une nouvelle implémentation sur un OS différent 
	peut elle-aussi apporter à la connaissance générale des détails importants.
	
	
