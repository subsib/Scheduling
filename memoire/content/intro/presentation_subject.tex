	
	\section{Introduction générale}
	Le monde du système embarqué est actuellement en plein foisonnement. 
	Tous ces systèmes n'ont cependant pas les mêmes contraintes. 
	Certains doivent respecter des échéances afin de garantir mathématiquement
	un fonctionnement conforme. Typiquement, on peut tolérer 
	un retard d'affichage de la page de son navigateur, il n'est cependant pas 
	concevable qu'un ABS de voiture soit en retard à cause d'une autre tâche 
	qui prendrait du de temps, mettant en péril la sécurité des passagers.
	Une proportion des systèmes embarqués est dite "en temps réel" et nécessite des 
	ordonnanceurs adaptés afin de prouver que les échéances seront toujours respectées.
	Ce champ de recherche -- temps réel -- a été largement nourri durant les vingt dernières années par de nombreuses publications scientifiques. \medskip
	
	Toutefois, il demeure un décalage important entre la connaissance scientifique et 
	la mise en pratique de celle-ci. 
	Par exemple, les systèmes embarqués disposent bien souvent de processeurs multi-c\oe{}urs.
	Cependant, leur gestion n'est à l'heure actuelle pas optimale, la plupart des systèmes sont  
	gérés soit comme des systèmes mono-processeurs, soit avec des algorithmes partitionnés \cite{paolillo_new_nodate}. 
	Dans le premier cas, les ressources ne sont pas utilisées de façon optimale, 
	et dans le second, des implémentations et tests ont montré empiriquement 
	que les algorithmes globaux pouvaient présenter des avantages intéressants, comme 
	une meilleure répartition de l'utilisation des processeurs \cite{baker_analysis_2005}. 
	\medskip	
	
	Si les systèmes embarqués n'ont pas toujours de grands 
	besoins en efficacité, un système dont l'efficacité n'est pas maximale n'utilise pas 
	ses ressources de façon optimale, ce qui conduit à du gaspillage. 
	Or, la gestion énergétique est un poste coûteux qui se doit d'être le plus réduit possible. 
	En l'occurrence, pour des installations comprenant beaucoup de systèmes embarqués, 
	comme les voitures ou les avions, cette dépense peut être substantielle. \medskip
	
	Le décalage entre la connaissance scientifique et les implémentations réelles 
	peut s'expliquer en partie par le fait qu'il soit compliqué de gérer le partage des ressources, 
	et que cette complexité n'apporte pas suffisamment d'avantages à l'heure qu'il est.\medskip
	
	Le fait que l'industrie n'implémente pas à ce jour de solutions plus \og performantes \fg{} 
	pour les systèmes embarqués pose plusieurs problèmes :
	\begin{enumerate}
		\item Le matériel n'est pas exploité de façon optimale.
		\item Par conséquent, la consommation en énergie des solutions déployées n'est 
		pas optimale.
		À l'heure actuelle, dans un monde où l'on cherche à consommer le moins possible, 
		cela pose question. Mais au delà, cela signifie également plus de maintenance 
		sur ces appareils parfois sans source de renouvellement d'énergie.
		\item Certains systèmes embarqués nécessitent une très basse consommation car 
		ils sont difficiles d'accès ou non faciles à recharger.
		\item Sur de grosses installations, comme des voitures ou un avions, cela peut représenter 
		un coût important en terme d'occupation de l'espace, de câblage entre différents systèmes, 
		consommation d'énergie, poids, achat de composants.
	\end{enumerate}
	
	Toutes ces raisons poussent à s'intéresser à une implémentation réelle et réaliste 
	d'ordonnanceurs connus dans la littérature, mais moins dans la réalité. Pour ce faire, c'est 
	l'ordonnanceur \textbf{UEDF} qui a été choisi et implémenté. 
	
Dans ce travail, on se propose d’analyser plusieurs éléments, outre d’exposer les choix d’implémentation et d’en expliquer les raisons : \medskip
	\begin{itemize}
		\item Comparer les performances attendues théoriquement et des résultats mesurés en pratique
		\item Comparer pour des mêmes ensembles de tâches les performances d’UEDF avec Global-EDF, tous les deux implémentés
		dans \textbf{HIPPEROS}
		\item Proposer des éléments afin de faciliter le passage de la littérature à la pratique, pour permettre
		ultérieurement à des chercheurs d'y travailler et rendre ces implémentations 
		faisables, voire moins fastidieuses.
	\end{itemize}
	
	\section{Choix de l'ordonnanceur}
	Le choix de l'ordonnanceur \textbf{UEDF} s'explique par plusieurs raisons :\medskip
	\begin{itemize}
		\item C'est un ordonnanceur global, qui minimise le nombre de processeurs utilisés et donc 
		optimise en théorie l'utilisation. Théoriquement, il minimise le nombre de 
		migrations.
		\item Son fonctionnement est particulier. Il prend par exemple en considération le \textbf{WCET} de la tâche 
		afin de calculer l'ordonnancement. Il n'a jamais été implémenté sur un véritable \textbf{RTOS}, 
		aussi cela constitue un résultat important de pouvoir affirmer s'il est implémentable 
		ou non (un simulateur ne remplace pas une implémentation réelle).
		\item La documentation autour de lui est détaillée et très bien écrite. Il apparaît que si des 
		difficultés à l'implémenter en découlent, cela ne pourra être reproché à la qualité du 
		travail initial.
	\end{itemize}
	
	\section{HIPPEROS}
	\customhighlight{HIPPEROS} (High Performance Parallel Embedded Real-time Operating Systems)
	est un \customhighlight{RTOS} (Real-Time Operating System) développé depuis plusieurs années par une spinoff de l'ULB.
	Il bénéficie des connaissances apportées par le monde de la recherche dans 
	le domaine des systèmes critiques avec multic\oe{}urs. Une de ses particularités 
	est sa modularité, qui permet d'adapter ses possibilités en fonction du système 
	lors de la compilation de l'OS, ainsi peut-on différencier principalement 
	deux installations en fonction des particularités.
	
	\customhighlight{HIPPEROS} est un candidat idéal pour l'implémentation d'un ordonnanceur 
	global. Il a cependant un fonctionnement propre qui pourra rendre l'implémentation 
	plus ou moins facile, et poser un certain nombre de problèmes. 
	En résumé, une nouvelle implémentation sur un OS différent 
	peut elle-aussi apporter à la connaissance générale des détails importants.
	