	\subsection{Tâche (temps réel)}
	Une tâche correspond à une sorte de programme, c'est-à-dire une série d'instructions 
	qui doivent être exécutées par un processeur. 
	Il y a plusieurs caractéristiques 
	et propriétés qui définissent une tâche, que nous allons définir ici : 
	\begin{itemize}
		\item \textbf{Temps de réalisation}\label{tempsderealisation} : C'est le moment $t$ où une tâche $\tau$ peut commencer à être exécutée.
		\item \textbf{Temps de réponse\label{Response Time}} : Temps maximum nécessaire à la réalisation de la tâche 
		entre le temps de réalisation et Fin.
		\item \textbf{Worst Case Execution Time}\label{wcet} : Afin de faciliter les calculs et l'abstraction du problème, l'on pose habituellement que le temps considéré d'exécution de 
		la tâche est le pire temps. Cela permet d'envisager le pire scénario, et ainsi 
		de s'assurer que si celui-ci est possible, alors dans de meilleures conditions, 
		la faisabilité est conservée. La notation utilisée dans le reste de ce document 
		est  \customhighlight{WCET} : \textbf{W}orst \textbf{C}ase \textbf{E}xecution \textbf{T}ime
		
		\item \textbf{Tâches périodiques} :
		Une tâche périodique est une tâche qui génère régulièrement des travaux.  
		Formellement, une tâche périodique est définie par un 4-uplet $(O, T, D, C)$ où \medskip
		\begin{itemize}
			\item \textbf{$O$} : est l'\og offset\fg{}, c'est-à-dire le temps que met la tâche à générer un premier travail.
			\item \textbf{$T$} : est la période, c'est-à-dire le temps qui sépare deux générations de travaux par la tâche. 
			Le premier travail est généré à l'instant $O$. Soit $i$, le nombre de fois que la tâche a été générée depuis le temps $0$,			
			 alors pour $i \in \mathbb{N}$, $\forall i \in \{0, \infty \}$,  $t = O + i\times T$ 
			où $t$ est le temps où la tâche est générée la $i$ème fois.
			\item \textbf{$D$} est l'échéance relative, c'est-à-dire le temps qui sépare au maximum la génération 
			d'un travail et sa réalisation.
			\item \textbf{$C$} est le temps de réalisation. Dans les algorithmes, on utilise -- comme expliqué précédemment -- le \customhighlight{WCET}.
		\end{itemize}	
		Dans le cas de tâches périodiques, on distingue trois cas différents : \medskip
		\begin{enumerate}
			\item si l'échéance est égale à la période $\forall \tau_i, D_i = T_i$ : tâche à \label{echeancesurrequete}échéance sur requête
			\item si l'échéance est inférieure ou égale à la période $\forall \tau_i, D_i \leq T_i $ , on dit que la tâche est à \label{echeancecontrainte} échéance contrainte
			\item s'il n'y a pas de contrainte particulière, la tâche est dite \og à échéance arbitraire\fg{}\label{echeancearbitraire}.
		\end{enumerate}
		Les solutions d'ordonnancement pour ces trois types de tâches diffèrent donc. 
		Globalement, on peut ordonner ces trois types de tâches : \medskip
		échéance sur requêtes $\subset$ échéance contrainte $\subset$ échéance arbitraire.
		
		\item \textbf{Laxité} : La laxité d'une tâche est la durée entre sa réalisation et son échéance. 
		Pour une tâche $\tau_i$, soit $D_i$ son échéance et $C_i$ son temps d'exécution, la laxité est définie comme :
		\vspace{-0.5cm}
		\begin{center}
			$L_i = D_i - C_i$
		\end{center}
		
		\item \textbf{Tâche sporadique} : Une tâche sporadique est une tâche qui génère de nouveaux travaux, 
		comme dans le cas de la tâche périodique. 
		La différence entre ces deux types est que la tâche sporadique 
		génère deux travaux avec un intervalle de temps au moins égal à la durée correspondant à sa période, 
		et pas exactement égal. Les systèmes embarqués ont souvent des tâches sporadiques à gérer : 
		en effet, un système qui est composé de capteur aura du mal à prévoir l'arrivée d'un signal de 
		celui-ci. 
		
		\item \textbf{Tâche apériodique} : Pour ce type de tâches, on ignore la régularité de 
		l'arrivée de nouveaux travaux. Les propriétés du travail ne sont connues que lorsqu'un travail est 
		généré dans le système.
		
		\item \textbf{Domaine système} : \todo{une bonne def doit se trouver}
		
	\end{itemize}
	
	\subsection{Migration}\label{migration}
	Une migration est le transfert d'une tâche ou d'un travail d’un processeur vers un autre.
	
	\subsection{Système de tâches}
	Un système de tâches et un ensemble de tâches devant être exécutées conjointement. 
	
	\subsection{Faisabilité}
	Un système de tâches est dit faisable si un ordonnancement existe.
	
	\subsection{Synchrone/asynchrone}
	Une tâche est définie par son offset, et lorsque toutes les tâches ont un \textit{offset} égal à 0, 
	elles sont dites \og à départ simultané\fg, et on parle de système synchrone. 
	Lorsque les \textit{offset} ne sont pas tous de 0, le système est dit à départs \og différés\fg{}, 
	on parle également de système asynchrone.
	
	\subsection{Hyperpériode}
	L'hyperpériode est un intervalle de temps qui est défini de façon formelle comme suit :\medskip
	$P = ppcm\{T_i\} \in \{1,... m\}\}$
	
	Cette notion est souvent utilisée afin d'obtenir l'intervalle de temps 
	minimal durant lequel tester l'ordonnançabilité d'un système par un ordonnanceur.
	
	\subsection{Travail}
	Un travail est une instance d'une tâche. 
	Il peut avoir des propriétés. 
	Dans certains algorithmes, certaines propriétés sont plutôt liées à la tâche, 
	et dans d'autres, ces mêmes propriétés sont plutôt dynamiques, et donc liées au travail.
	\begin{itemize}
		\item \textbf{Début, Fin} : Le moment où un travail commence à être exécuté, ou termine son exécution.
		\item \textbf{Échéance, temps limite} : Correspond au temps limite au delà duquel le travail doit avoir été exécuté. Il existe deux types d'échéances : 
		une absolue, et une relative. L'échéances relative est une valeur fixe qui 
		est une propriété de la tâche, elle dépend donc du temps de réalisation.\medskip
		L'échéance absolue quant à elle est calculée avant l'exécution, et correspond 
		à un temps $t$ absolu, elle est donc une propriété du travail.
	\end{itemize}
	
	\subsection{RTOS}
	(Real-Time Operating System) Système d'Exploitation Temps Réel.\medskip
	Un système d'exploitation Temps Réel un système d'exploitation implémenté pour les systèmes 
	à temps réel, c'est-à-dire dont l'objectif est d'assurer le respect de certaines échéances. 
	
	Cela concerne pour beaucoup les systèmes dits \og critiques\fg{} , c'est-à-dire dont la fiabilité 
	est primordiale (avions, centrales nucléaires, pacemakers, etc.). Ce type de dispositifs 
	est actuellement très répandu, notamment dans les voitures.
	
	
	Les contraintes sont différentes de celles des systèmes d'exploitations des ordinateurs 
	de bureau puisque l'on doit garantir le respect des échéances associées à chacune des tâches. 
	
	Cependant, si les systèmes d'exploitation temps réel ont des besoins de respect des échéances, 
	cela ne signifie pas qu'ils aient forcément de grands besoins en efficacité ou en rapidité, ce qui 
	est primordial est que les échéances soient respectées. 
	En effet, dans un système dit \og critique\fg{} , une défaillance du système pourrait entraîner des dégâts 
	très importants. 
	Dans ce cadre, l'on a besoin de garanties que le système soit fiable, et c'est là une priorité absolue.
	L'efficacité en elle-même n'est pas le besoin le plus fondamental de ce type de système.\medskip
	
	\subsection{Contraintes strictes, contraintes relatives} 
	Dans le cas strict, le système doit impérativement respecter tous les temps limites. Aucun dépassement n'est toléré. Dans le cas des contraintes relatives, ce respect 
	est moins impératif, et on pourra dépasser les délais occasionnellement. \medskip
	Dans la littérature scientifique, on parle de \customhighlight{hard real-time} dans le cas d'un ordonnanceur 
	qui ne tolère aucun dépassement, et de \customhighlight{soft real-time} dans le cas contraire.
	
	\subsection{Ordonnanceur}
	Un ordonnanceur (\textit{Scheduler}) est la partie logicielle de 
	l'\textit{OS} chargée d'orchestrer l'ordre d'exécution des tâches du système 
	selon des priorités fixées à l'avance ou durant l'exécution. 
	On distingue trois types d'assignation de priorités des ordonnanceurs :
	
	
	\begin{enumerate}
		\item Priorité fixée au niveau des tâches : Ce type d'ordonnanceur fixe la priorité 
		avant l'exécution, et celle-ci dépend donc des attributs de la tâche. 
		Deux exemples d'ordonnanceurs de ce type seront présentés plus loin : Rate Monotonic et 
		Deadline Monotonic.
		\item Priorité fixe au niveau des travaux : La priorité est fixée par l'ordonnanceur à l'arrivée du travail 
		dans l'exécution, et elle ne peut pas changer jusqu'à la réalisation du travail. Un exemple sera présenté plus loin : Earliest Deadline First.
		\item Priorité dynamique : L'ordonnanceur peut recalculer à tout moment de l'exécution 
		la priorité du travail avec un exemple connu, Least Laxity First.
	\end{enumerate}
	
	Selon les cas, l'ordonnanceur peut avoir à gérer un seul processeur. Dans ce 
	document, on parlera dans ce cas d'ordonnanceur mono-processeur. Toutefois, de plus en 
	plus de systèmes possèdent plusieurs processeurs, et il existe deux grandes familles 
	d'ordonnanceurs associés à ces systèmes : les partitionnés, ou les globaux. \medskip
	
	\subsection{Optimalité}\label{optimal}
	Un ordonnanceur est dit optimal si pour un système de tâche pour lequel un ordonnancement 
	existe, l'ordonnanceur permet de l'ordonnancer.
	
	\subsection{Économe} 
	Un ordonnanceur peut-être \og économe\fg{} , ou \og non-économe\fg{} . Dans le premier cas, si le processeur est libre, 
	et qu'une tâche est prête à être exécutée, l'ordonnanceur donnera l'instance : cela signifie 
	qu'on évite les temps d'inactivité du processeur. Dans le second cas, une tâche ne sera pas toujours ordonnancée.
	Dans le reste de ce document, les ordonnanceurs présentés sont tous économes selon cette définition.
	
	\subsection{En ligne, hors ligne}
	Un ordonnanceur peut faire ses opérations d'ordonnancement de deux façons : \medskip
	\begin{itemize}
		\item Durant le runtime, on parlera d'opérations en ligne
		\item Avant l'exécution, on parlera d'opérations hors ligne
	\end{itemize}	
	
	\subsubsection{Ordonnanceur multi-processeur Partitionné}
	Un ordonnanceur partitionné composé de $n$ tâches et de $m$ processeurs divise 
	le système en $m$ sous-systèmes qui seront ensuite ordonnancés par autant d'ordonnanceurs 
	mono-processeurs. Le problème principal consiste à découper efficacement le système de tâches.
	
	\subsubsection{Ordonnanceur multi-processeur Global}
	Un ordonnanceur global ne sépare pas le système en sous-systèmes mais gère une 
	queue de tâches prêtes à être exécutées. Ainsi à chaque instant où il devra prendre une décision 
	d'ordonnancement, il devra ordonnancer toute la liste des $m$ processeurs.\medskip
	Contrairement aux ordonnanceurs partitionnés, les migrations [\ref{migration}] sont possibles. Le problème 
	est donc de réduire leur nombre pour ne pas faire de mouvements inutilement.
	
	\subsection{Clairvoyance}
	Un ordonnanceur est dit \og clairvoyant\fg{}  s'il peut prédire des éléments (le début d'un travail) des tâches du système qu'il ordonnance.
	
	\subsection{Préemption} 
	Un système est dit préemptif s'il a la capacité 
	de mettre l'exécution d'un travail en pause et d'exécuter un autre à la place. 
	
	\subsection{Utilisation}
	Le facteur d'utilisation $U_i$ d'une tâche périodique $\tau_i$ est le rapport entre 
	son temps d'exécution et sa période. Par exemple, pour une tâche $\tau_i$ tel que 
	$WCET = 30 ms$ et $T = 50 ms$, le processeur doit allouer $\frac{30}{50}$ du temps 
	total d'exécution à cette tâche.\medskip
	
	L'utilisation (ou le facteur d'utilisation) d'un système périodique est la proportion de temps 
	passée par le processeur à l'exécution de tâches d'un système donné. En pratique, l'utilisation 
	est calculée avec le WCET, qui est la borne supérieure du temps d'exécution réel, on 
	peut donc penser que l'utilisation réelle sera inférieure à ce nombre.
	Le calcul de l'utilisation permet dans certains cas et certaines conditions de donner une indication 
	à propos de la faisabilité du système par un certain ordonnanceur. 
	\medskip
	
	Liu et Layland \cite{liu_scheduling_1973} en donnent une définition formelle.\medskip 
	
	Soit $C_i$ le temps d'exécution d'une tâche $\tau_i$, et soit $T_i$ sa période, voici la définition de 
	l'utilisation pour un système de $m$ tâches :\medskip
	\begin{center}
		$U_{tot} = \sum_{i = 1}^{m}(\frac{C_i}{T_i})$
	\end{center}
	
	\subsection{Laxité}
	\label{laxite}
	La laxité d'un travail est définie par la différence entre le temps de réponse et l'échéance 
	d'une tâche. Sa définition formelle varie en fonction de l'ordonnanceur dont il est question.
	
	\subsection{Instant critique}
	\label{instantcritique}
	Un instant critique est le moment où le temps de réponse d'une tâche est le plus long.
