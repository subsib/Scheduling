\begin{abstract}
		Le problème de l'ordonnancement des tâches dans un système d'exploitation est 
		central : il a un impact majeur dans la gestion des ressources. 
		En effet, même si les applications sont sobres individuellement, dans le cas où 
		l'ordonnanceur n'offre pas une utilisation optimisée des ressources, alors 
		il en résultera un gaspillage tant d'énergie que d'infrastructure.    
		Malgré l'existence d'ordonnanceurs multi-processeurs globaux optimaux décrits 
		dans la littérature scientifique, l'industrie continue à ce jour à préférer des 
		solutions mono-processeurs voire multi-processeurs partitionnées plus simples, 
		éprouvées mais moins efficaces. 
		Ainsi, les conditions, les contraintes, et les 
		effets de la mise en œuvre pratique de tels ordonnanceurs restent mal connus. 
		
		Le présent document expose le choix d'un ordonnanceur : \textbf{UEDF}, 
		son implémentation dans un RTOS : \textbf{HIPPEROS} ainsi qu'une évaluation de 
		ses performances, en le comparant à un autre algorithme connu : Global-EDF, avant de proposer des améliorations.\newline
		
		Nous prouvons avec ce travail qu'\textbf{UEDF} est implémentable, 
		présente des caractéristiques intéressantes mais nécessite de l'optimisation 
		avant de pouvoir éventuellement montrer sa supériorité sur \textbf{Global-EDF}.
		
		
\end{abstract}