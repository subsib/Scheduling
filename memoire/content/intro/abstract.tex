\begin{abstract}
		Le problème de l'ordonnancement des tâches dans un système d'exploitation est 
		central : il a un impact majeur dans la gestion des ressources. 
		En effet, même si les applications sont sobres individuellement, dans le cas où 
		l'ordonnanceur n'offre pas une utilisation optimisée des ressources, alors 
		il en résultera un gaspillage tant d'énergie que d'infrastructure.    
		Malgré l'existence d'ordonnanceurs multi-processeurs globaux optimaux décrits 
		dans la littérature scientifique, l'industrie continue à ce jour à préférer des 
		solutions mono-processeurs voire multi-processeurs partitionnées plus simples, 
		éprouvées mais moins efficaces. 
		Ainsi, les conditions, les contraintes, et les 
		effets de la mise en œuvre pratique de tels ordonnanceurs restent mal connus. 
		
		Le présent document expose cette problématique, puis propose un état de l'art 
		embrassant l'éventail des ordonnanceurs mono-processeurs, multiprocesseurs 
		partitionnés jusqu'aux multi-processeurs globaux connus. 
		
		Le projet final étant d'implémenter un ordonnanceur au sein du système 
		d'exploitation temps réel \textbf{HIPPEROS}, pour analyser ses performances 
		en situation réelle, cet état de l'art permet de motiver le choix de 
		l'ordonnanceur étudié : il s'agira de \textbf{UEDF}.
\end{abstract}