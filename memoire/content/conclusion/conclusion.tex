	Nous avons exploré une partie de la littérature scientifique au sujet des ordonnanceurs 
	globaux ou semi-partitionnés afin d'en sélectionner un pour implémentation et tests. 
	Il ressort de cette étude que plusieurs candidats présentent des intérêts, sont mieux 
	connus dans la littérature que dans la pratique, et gagneraient à être mieux analysés. 
	Notre choix sera influencé par ces paramètres : \medskip
	\begin{itemize}
		\item L'ordonnanceur est-il optimal ?
		\item Quels systèmes de tâche peut-il ordonnancer ?
		\item Des efforts ont-ils été fournis afin d'éviter les migrations ?
		\item A-t-il déjà bénéficié d'implémentations ?
	\end{itemize}
	De ces paramètres, il ressort que l'algorithme \textit{U-EDF} est optimal pour la classe périodique, 
	semi-partitionné, peut également ordonnancer les systèmes sporadiques à échéances arbitraires. 
	
	En outre, G. Nelissen est accessible par e-mail, intéressé par une implémentation réelle, 
	et son travail montre une très bonne connaissance des ordonnanceurs et des systèmes d'exploitation. En 
	effet, dans sa thèse, on trouve une progression qui mène à une implémentation qui soit faisable. 
	\todo{developper : il fait gaffe à ce que ce soit faisable, et on voit, quand on le lit, qu'il 
	comprend les enjeux, le fonctionnement d'un RTOS}
	
	C'est donc vers cet algorithme que notre choix se porte.
