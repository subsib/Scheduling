L'objectif initial de ce travail était avant tout de choisir un ordonnanceur 
issu de la littérature scientifique n'ayant bénéficié d'aucune implémentation dans un véritable RTOS. 
Au terme d'une revue de l'état de l'art sur les ordonnanceurs actuels,
nous avons choisi de développer une implémentation d'\textbf{UEDF}, un algorithme qui a été élaboré par une 
équipe de chercheurs de l'\textbf{ULB}, pour ses propriétés mais également 
car la communication autour de lui nous semblait accessible.\newline

Au terme d'une première implémentation dans \textbf{HIPPEROS}, nous pouvons déjà affirmer 
qu'aucun élément ne rend cet algorithme non implémentable dans cet RTOS.\newline

Par ailleurs, au terme des tests que nous avons fait afin de mesurer ses comportements, nous pouvons dire :
\begin{itemize}
	\item sa répartition de charge dans le cas d'ensembles de $m \times 100$ \% n'est pas mauvaise, mais le dernier cœur 
	n'a pas autant de charge que les autres. Ceci devrait être amélioré en apportant quelques modifications, et nous 
	pensons que rien ne s'oppose à ce que ce résultat s'améliore grandement. 
	\item Des optimisations telles que de maintenir un ordonnancement virtuel afin d'éviter de nombreuses migrations sont 
	indispensables, sinon l'algorithme perd de son intérêt, et ce, même face à un algorithme décrit comme n'étant pas 
	optimal dans la littérature. En d'autres termes, cela fait perdre l'optimalité d'\textbf{UEDF} de procéder 
	à son implémentation \og{}naïve\fg{}. 
	\item L'algorithme détermine ses décisions en se basant sur le WCET de la tâche, or, cette valeur dépend elle-même de 
	l'algorithme d'ordonnancement. Cela comporte des avantages, néanmoins rend le calibrage des ensembles délicat. 
\end{itemize}

Au terme de ce travail, nous pensons plus largement à la littérature scientifique décrivant d'autres algorithmes. 
Le travail d'implémentation s'est avéré complexe pour bien des raisons, et si une équipe souhaite que son algorithme 
bénéficie d'une implémentation dans un véritable RTOS, il nous semble absolument indispensable 
que les papiers dépassent le stade de la preuve mathématique. 
Le monde de l'industrie comme des sciences bénéficierait de multiples travaux de ce genre 
afin d'augmenter la confiance en des solutions théoriques non éprouvées.
\newline
\newpage