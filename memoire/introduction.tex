% chapitre introduction

% TODO trouver meilleur nom qu'introduction, l'introduction est une partie de cette première partie.
\chapter{Introduction}
\setcounter{page}{1}
\begin{quotation}
	\noindent ``\emph{certains aiment insérer une ou des citations particulières, en rapport avec le chapitre.}''
	\begin{flushright}\textbf{auteur, date}\end{flushright}
\end{quotation}

\vspace*{0.5cm}

\section{Contexte et objectifs du mémoire}
%Blabla
%
%Pour une introduction rapide mais quand même bien détaillée de \LaTeX et \LaTeX2e, vous pouvez consulter, outre les bouquins classiques \cite{lamp,mittel}, le site \cite{oetik}.
%
%Vous aurez peut-être besoin d'utiliser d'autres packages que ceux mentionnés au début du fichier! Pour écrire des algorithmes, consultez par exemple le site \cite{fiorio}.

Dans le paysage actuel de l'informatique, 

\section{Quelques sections bien choisies}
Blabla

Blabla
\section{Structure de ce document}
%Dans le chapitre \ref{chap2}
%Blabla

\section{Contributions du mémoire}
\noindent
La liste suivante présente nos principales contributions :
\vspace{1cm}
\begin{enumerate}
	\item j'espère qu'il y en aura beaucoup~(Chapitres~\ref{chap2} et untels).
	\item et encore~(Section~\ref{sec-untel}).
	\item et encore~(Sections untel, untel et untel).
\end{enumerate}

\clearpage
\section{Notations}

%
Il peut être intéressant de rassembler ici les principales notations, leur signification intuitive, et la page où c'est défini plus en détail ou une référence bibliographique si cela s'impose.
